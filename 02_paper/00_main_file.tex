\documentclass[oneside, english, reqno, 11pt, headsepline, BCOR=8.5mm]{scrbook} %BCOR8.5mm, ohne DIV openany % exchange 'article' with 'scrbook' % oneside, english, reqno, 11pt, headsepline, BCOR=8.5mm
\usepackage[T1]{fontenc}
\usepackage[utf8]{inputenc}
\usepackage{lmodern}
%\setlength{\parskip}{\smallskipamount}
%\setlength{\parindent}{0pt}
%\usepackage{textcomp}
\usepackage{amsmath, amssymb, amsthm, amstext,mathrsfs, mathtools}
\newcommand\norm[1]{\left\lVert#1\right\rVert} % define adjustable norm
\usepackage{esint}
\usepackage{babel}
\usepackage{microtype}
\usepackage{relsize} % for \mathlarger
\usepackage[english=usenglishmax]{hyphsubst}	
\usepackage{booktabs}
\usepackage{scrlayer-scrpage}
\usepackage{graphicx}
%\usepackage{ngerman}
\usepackage[babel,german=quotes]{csquotes}
\usepackage{bibentry}
\usepackage{tabularx}	
\usepackage{hhline} % for double h line
\usepackage{commath}			% for \norm{...}
\usepackage{setspace, natbib}
\usepackage{bm}
\usepackage{threeparttable}
\usepackage{comment}\includecomment{notthisone}
\usepackage[colorinlistoftodos]{todonotes}
\usepackage{enumerate}
\usepackage{epstopdf} 
\usepackage{dsfont}
\usepackage{pstool}
\usepackage{subfig}
\usepackage{natbib}
\usepackage{hyperref}
\usepackage{scalerel} % for the \scaleto command
\usepackage{array, booktabs}
\usepackage{bbm}
\usepackage{setspace}
\usepackage{mathcomp}
\usepackage{multirow}
\usepackage[toc,page]{appendix}
\usepackage{lipsum}
\usepackage{titling} % for abstract
%\usepackage[top=3.5cm,bottom=4.5cm]{geometry}
\usepackage{verbatim} % to do a block comment
\usepackage{float} % figures tsay where they should be with "H"
\usepackage{dsfont} % for indicator function
\usepackage{fancyvrb} % for "\Verb" macro
\VerbatimFootnotes    % enable use of \Verb in footnotes

\renewcommand\thesubtable{(\alph{subtable})}

\DeclareOldFontCommand{\rm}{\normalfont\rmfamily}{\mathrm}
\DeclareOldFontCommand{\sf}{\normalfont\sffamily}{\mathsf}
\DeclareOldFontCommand{\tt}{\normalfont\ttfamily}{\mathtt}
\DeclareOldFontCommand{\bf}{\normalfont\bfseries}{\mathbf}
\DeclareOldFontCommand{\it}{\normalfont\itshape}{\mathit}

%\usepackage{algorithm}
%\usepackage{algcompatible}
%\usepackage{algorithmicx}
\usepackage[ruled]{algorithm2e}
%\usepackage[noend]{algpseudocode}

%\floatname{algorithm}{Algorithm}
%\renewcommand{\algorithmicrequire}{\textbf{Input:}}
%\renewcommand{\algorithmicensure}{\textbf{Output:}}

\usepackage{mathtools}
%\mathtoolsset{showonlyrefs} % activate this for no equation numbering

\makeatletter
\def\BState{\State\hskip-\ALG@thistlm}
\makeatother

\input{style/ee.sty} 


%\usepackage{cite} %Literaturverwaltung 
%\usepackage{setspace} % Zeilenabstand aendern

%\setcounter{tocdepth}{3}

%\bibliographystyle{plain}
\pagestyle{scrheadings}%{plain}%
\makeatletter

\newtheoremstyle{Definition}
  {0.2cm}                   %Space above
  {0.2cm}                   %Space below
  {\normalfont}           %Body font
  {}                      %Indent amount (empty = no indent,
                          %\parindent = para indent)
  {\bfseries}  						%Thm head font
  {.}                     %Punctuation after thm head
  { }              				%Space after thm head: " " = normal interword
                          %space; \newline = linebreak
  {}
                          %Thm head spec (can be left empty, meaning
                          %`normal')
                          
\newtheoremstyle{Theorem}
  {0.2cm}                   %Space above
  {0.2cm}                   %Space below
  {\itshape}           		%Body font
  {}                      %Indent amount (empty = no indent,
                          %\parindent = para indent)
  {\bfseries}  						%Thm head font
  {.}                     %Punctuation after thm head
  { }              				%Space after thm head: " " = normal interword
                          %space; \newline = linebreak
  {}
                          %Thm head spec (can be left empty, meaning
                          %`normal')


%%%%%%%%%%%%%%%%%%%%%%%%%%%%%% Textclass specific LaTeX commands.
\numberwithin{equation}{section}
\numberwithin{figure}{section}


\theoremstyle{Theorem}
	\newtheorem{cor}{Corollary}[chapter]
	\newtheorem{prop}{Proposition}[chapter]
	\newtheorem{lem}{Lemma}[chapter]
	\newtheorem{thm}{Theorem}[chapter]
	\newtheorem{thmA}{Theorem A}[chapter]
	\newtheorem{lemA}{Lemma A}[chapter]

\theoremstyle{Definition}
	\newtheorem{example}{Example}[chapter]
	\newtheorem{rem}{Remark}[chapter]
	\newtheorem{defn}{Definition}[chapter]
    \newtheorem{assump}{Assumption}[chapter]
	
\newcommand\Chapter[2]{\chapter
  [#1\hfil\hbox{}\protect\linebreak{\itshape#2}]%
  {#1\\[2ex]\itshape#2}%
}

\renewcommand\topfraction{0.85}
\renewcommand\bottomfraction{0.85}
\renewcommand\textfraction{0.1}
\renewcommand\floatpagefraction{0.85}

\setlength{\footskip}{2cm}
\setlength{\parindent}{2em}
\setlength{\parskip}{.5em}

\newcommand{\Levy}{L\'{e}vy }


\allowdisplaybreaks

\raggedbottom

\makeatother

%%%%%%%%%%%%%%%%%%%%%%%%%%%%%%%%%%%%%%%%%%%%%%%%%%%%%%%%%%%
% just some information for \maketitle
\title{Shrinkage Bayesian Causal Forest with Instrumental Variable}
\subtitle{Here should be a fancy subtitle}
\author
{Jens Klenke$^{1,\barwedge}$, Lennard Ma{\ss}mann$^{1,2,\ast}$ \\
\\
\normalsize{$^{1}$Chair of Econometrics, University of Duisburg-Essen}\\
\normalsize{$^{2}$Ruhr Graduate School in Economics}\\
\\
\normalsize{$^\ast$E-mail:  lennard.massmann@uni-due.de}\\
\normalsize{$^\barwedge$E-mail:  jens.klenke@vwl.uni-due.de}\\
}
\date{\today}
%\publishers{Aperture Science Enrichment Center}

%%%%%%%%%%%%%%%%%%%%%%%%%%%%%%%%%%%%%%%%%%%%%%%%%%%%%%%%%%%

%%%%%%%% to introduce an abstract %%%%%%%%%
%   Reduce the margin of the summary:
\def\changemargin#1#2{\list{}{\rightmargin#2\leftmargin#1}\item[]}
\let\endchangemargin=\endlist 

%   Generate the environment for the abstract:
\newcommand\summaryname{Abstract}
\newenvironment{Abstract}%
    {\small\begin{center}%
    \bfseries{\summaryname} \end{center}}
%%%%%%%% to introduce an abstract %%%%%%%%%



\newcommand{\Rbb}{\mathbb{R}}
\newcommand{\Nbb}{\mathbb{N}}
\newcommand{\Zbb}{\mathbb{Z}}
\newcommand{\Qbb}{\mathbb{Q}}
\newcommand{\Ebb}{\mathbb{E}}
\newcommand{\ind}{\perp\!\!\!\!\perp} % statistical independence sign


%%%%%%%% todo comments %%%%%%%%%

\newcommand{\todoComment}[2][]{\todo[linecolor=red!20,backgroundcolor=red!20,bordercolor=red!20, size=\tiny,#1]{#2}}



\begin{document}

%\onehalfspacing % 1.5-facher Zeilenabstand
\frontmatter
%\input{title}
\maketitle

%\listoftodos

\cleardoublepage


\begin{Abstract}
\begin{changemargin}{1cm}{1cm}
Your text:\lipsum[10]
\end{changemargin}
\end{Abstract}

\cleardoublepage

\pagenumbering{Roman}

% \begin{center}
% \textbf{Frequently used symbols and abbreviations:}
% \end{center}
% \begin{eqnarray*}
% \mathcal{S}^{tr} &:& \textup{training sample (for tree-growing process)} \\
% \mathcal{S}^{est} &:& \textup{estimation sample (for leaf-wise estimation)} \\
% \mathcal{S}^{te} &:& \textup{test sample} \\
% \Pi &:& \textup{tree/partitioning of feature space } \mathbb{X}\\
% MSE &:& \textup{mean squared error} \\
% \mu(x; \mathcal{S}, \Pi) &:& \textup{conditional mean function} \\
% \tau(x; \mathcal{S}, \Pi) &:& \textup{conditional average treatment effect} \\
% \tau_i &:& \textup{unit-level treatment effect} \\
% X_i &:& \textup{individual $i$'s feature matrix of dimension } N \times K \\
% D_i &:& \textup{individual $i$'s binary treatment indicator, $D_i \in (0,1)$} \\
% Y_i(D_i) &:& \textup{individual $i$'s potential outcome}\\
% Y_i^{obs} &:& \textup{individual $i$'s observed potential outcome}\\
% \mathbb{E} &:& \textup{expectation operator}\\
% Pr(.) &:& \textup{probability}\\
% Pr(.|.) &:& \textup{conditional probability}\\
% \hspace{1cm} \\
% \textup{to be continued ...}
% \end{eqnarray*}


\begin{comment}
\mathbb{N} &:& \{1,2,\ldots\}\\
\mathbb{R} &:& \textup{set of real numbers}\\
%\mathbb{R}_+ &:& (0,\infty)\\
\mathbb{C} &:& \textup{set of complex numbers}\\
\mathrm{i} &:& \textup{imaginary unit, }  \mathrm{i}^2=-1\\
\mathrm{Re}(z) &:& \textup{real part } a \textup{ of } z=a+b\mathrm{i}\in\mathbb{C}\\
\arg(z) &:& \textup{the argument of a complex number } z\\
A^{\mathrm{T}} &:& \textup{transpose of a matrix } A\in\mathbb{R}^{m\times n}\\
\left\langle x,y\right\rangle &:& x^{\mathrm{T}}y, \textup{ for } x,y\in\mathbb{R}^n\\
\mathds{1}_A &:& \textup{indicator function } \mathds{1}_A(x)=1, \textup{ if } x\in A;\ \mathds{1}_A(x)=0, \textup{ if } x\notin A\\
\Gamma(x) &:& \textup{gamma function for } x>0\\
J_{\lambda}(x) &:& \textup{Bessel function of the first kind}\\
Y_{\lambda}(x) &:& \textup{Bessel function of the second kind}\\
K_{\lambda}(x) &:& \textup{modified Bessel function of the second kind}\\
f(x)\sim g(x) &:& \lim_{x\to a}\frac{f(x)}{g(x)}=1, a\in\mathbb{R}\cup\{\pm\infty\}\\
f(x)=\mathcal{O}( g(x)) &:& \limsup_{x\to a}\frac{|f(x)|}{|g(x)|}<\infty, a\in\mathbb{R}\cup\{\pm\infty\}\\
X\sim \mu &:& X \textup{ has distribution } \mu\\
\stackrel{\mathcal{L}}{=} &:& \textup{equality in distribution}\\
\stackrel{\mathcal{L}}{\rightarrow} &:& \textup{convergence in distribution}\\
X_n\stackrel{\mathcal{L}}{\sim}Y_n &:& X_n\stackrel{\mathcal{L}}{\rightarrow}X,Y_n\stackrel{\mathcal{L}}{\rightarrow}X\\
X_n=o_P(1) &:& X_n\to0 \textup{ in probability for } n\to\infty\\
\textup{a.s.} &:& \textup{almost surely}\\
\textup{i.i.d.} &:& \textup{independent, identically distributed}\\
N(\mu,\sigma^2) &:& \textup{normal distribution with mean } \mu \textup{ and variance } \sigma^2\\
t(\nu,\mu,\sigma^2) &:& \textup{Student } t \textup{ distribution with degree of freedom } \nu, \textup{ location } \mu \textup{ and scale } \sigma\\
t(\nu,\mu,\sigma^2,\beta) &:& \textup{Skew Student } t \textup{ distribution with skewness parameter } \beta\\
R\Gamma(\alpha,\beta) &:& \textup{inverse gamma distribution with shape parameter } \alpha \textup{ and scale } \beta\\
Poi(\lambda) &:& \textup{Poisson distribution with intensity parameter } \lambda\\
Exp(\lambda) &:& \textup{Exponential distribution with rate } \lambda\\
\mathcal{U}_{[a,b]} &:& \textup{Uniform distribution on } [a,b] 
\end{comment}


%\newpage{}

%\thispagestyle{empty}
\tableofcontents
%\newpage{}


\mainmatter
%\pagenumbering{arabic}
\microtypesetup{activate=true}


\chapter{Introduction}

\cite{hahn_bayesian_2020} 

\section{Literature}


\begin{itemize}
    \item Using machine learning to infer heterogeneous effects in observational studies often focuses on CATE estimation under regular assignment mechanisms 
    %($athey_generalized_2019$)
    \item Focus in this work: Methods to discover heterogeneous effects in the presence of imperfect compliance
    \item Methods
    \begin{itemize}
        \item tree-based 
        %($bargagli_stoffi_causal_2020$)
        %<!-- (perform worse compared to ensembles) -->
        \item ensemble-of-trees 
        %($@athey_generalized_2019$)
        %<!-- (rely on large samples, hart to interpret, Hahn 2019, Wendling 2018) -->
        \item deep-learning-based methods 
        %($@hartford_counterfactual_2016$)
        %<!-- (computationally intensive, hyper-parameter-sensitive, intolerant against unmeasured variation, hart to interpret) -->
    \end{itemize}
    \item BCF-IV: Discovers and estimates HTE in an interpretable way %($@bargagli-stoffi_heterogeneous_2020$)
    \begin{itemize}
        \item BCF: BART-based semi-parametric Bayesian regression model, able to estimate HTE in regular assignment mechanisms, even with strong confounding %($@hahn_bayesian_2020$)
        \item Use BCF to estimate $\hat\tau_C(x)$ and  $\widehat{ITT}_{Y}(x)$ such that the conditional Complier Average Causal Effect $\hat\tau^{cace}(x) = \frac{\widehat{ITT}_{Y}(x)}{\hat\tau_C(x)}$
    \end{itemize}
\end{itemize}



\section{Contribution}

\begin{itemize}
    \item - BART benefits %(@linero_softbart_2022):
    \begin{itemize}
        \item good performance in high-noise settings 
        \item shrinkage to/emphasize on low-order interactions
        \item established software implementations (`BayesTree`, `bartMachine`, `dbarts`)
    \end{itemize}  
    \item BART shortcomings %(@linero_softbart_2022, @hahn_bayesian_2020):
    \begin{itemize}
        \item non-smooth predictions as BART prior produces stepwise-continuous functions
        \item BART prior is overconfident in regions with weak common support
    \end{itemize}
    \item Research proposal: Rewrite the BCF-IV model with SoftBART instead of BART prior to account for sparsity
\end{itemize}


\chapter{Potential outcomes and irregular assignmnet}

We follow the setup of \cite{bargagli_stoffi_causal_2020} and \cite{bargagli-stoffi_heterogeneous_2022} who follow the usual notation of the Rubin's causal model. 

Given a set of $N$ indiviudals, indexed by $i = 1, \ldots, N$, we denote with $Y_i$ a generic outcome variable, with $W_i$ a binary treatment indicator, with $X$ an $N \times P$ matrix of $P$ control variables, and with $X_i$ the $i$-th $P$-dimensional row vector of covariates.
Let us define the pair of potential outcomes $Y_i(W_i)$ by using the Stable Unit Treatment Value Assumption (SUTVA). 
\begin{assump}{Stable unit treatment value assumption (SUTVA).}
   \label{assump:SUTVA} \\
   \begin{align*}
   \label{SUTVA}
   \text{If } W_i = w \text{, then } Y_i(w) = Y_i^{obs} \text{ , } \forall w \in \{0, 1 \} \text{ , } \forall i \in \{1, ..., N \}. 
   \end{align*}
\end{assump}
Therefore, it holds that $Y_i(W_i = 1) = Y_i(1)$ for an indiviudal $i$ under assignment to the treatment group and $Y_i(W_i = 0) = Y_i(0)$ under assignment to control group.
Despite one is unable to observe both potential outcomes at the same time for each individual one is able to observe the potential outcome that aligns with the assigned treatment status such that

\[
Y_i^{\text{obs}} = Y_i(1)W_i + Y_i(0)(1 - W_i).
\]

Under the strong ignorability assumptions of \ref{assump:unconfoundedness} and \ref{assump:overlap} we operate under the regular assignment mechanism which, through \ref{assump:unconfoundedness}, prevents the existence of unmeasured confounding and, by \ref{assump:overlap}, allows for unbiased treatment effect estimation in support of the covariate space. 

\begin{assump}{Unconfoundedness.}
   \label{assump:unconfoundedness} \\
   \begin{align*}
   W_i \ind \left(Y_i(1), Y_i(0)\right) |X_i), \text{ or equivalently, } \\
   Pr(W_i | (Y_i(1), Y_i(0), X_i) = Pr(W_i | X_i).   
   \end{align*}
\end{assump}

\begin{assump}{Positivity.}
   \label{assump:overlap} 
   \begin{align*}
   \epsilon < p(X_i=x) < 1 - \epsilon \text{ with probability 1, } \forall   \epsilon>0 , \forall x \text{ in support of } X_i.
   \end{align*}
\end{assump}

 
Using strong ignorability, one can define the CATE to analyse heterogeneous treatment effects by

\begin{defn}{Conditional Average Treatmnet Effect (CATE).}
   \label{defn:CATE} \\
   \begin{align*}
      \tau(x) = \mathbb{E}[Y_i^{\text{obs}} \mid W_i = 1, X_i = x] - \mathbb{E}[Y_i^{\text{obs}} \mid W_i = 0, X_i = x].   
   \end{align*}
\end{defn}

In this work, we deviate from this regular assignment mechanism that is implied by assumptions \ref{assump:unconfoundedness} and \ref{assump:overlap} and operate under the scenario of an irregular assignment mechanism.
This irregular assignment mechanism allows for a violation of compliance between (quasi-)randomized treatment assignment, $Z_i$, and treatment receipt, $W_i$, such that just the assignment to treatment is assumed to be unconfounded through a valid choice of $Z_i$ but the treatment receipt might be confounded.
A remedy to the issue of confoundedness in treatment receipt is the Instrumental Variable (IV) approach.
A proper instrumental variable, $Z_i$, affects treatment receipt, $W_i$, while not being allowed to affect $Y_i$ directly such that treatment receipt depends on the treatment assignment by $W_i(Z_i)$.
Based on the functional relation between $W_i$ and $Z_i$ and in case of a binary treatment variable, one can define four subgroups of individuals, $G_i$, such that  
\begin{defn}{Subgroups $G_i$.}
   \label{defn:subgroups}
   \begin{align*}
      G_i=
      \begin{cases}
         C,& \text{if } W_i(Z_i = 0) = 0, W_i(Z_i = 1) = 1 \\
         D,& \text{if } W_i(Z_i = 0) = 1, W_i(Z_i = 1) = 0 \\
         AT,& \text{if } W_i(Z_i = 0) = 1, W_i(Z_i = 1) = 1 \\
         NT,& \text{if }W_i(Z_i = 0) = 0, W_i(Z_i = 1) = 0
      \end{cases}.
   \end{align*}
\end{defn}
where $C$, $D$, $AT$ and $NT$ are abbreviations for Compliers, Defiers, Always-Takers, Never-Takers. 
The proportion of individuals that belong to each subgroup is defined as $\pi_{G_i}$, i.e. the proportion of compliers reads $\pi_{C}$.
Considering the distinction between $Z_i$ and $W_i$, the Intention-To-Treat (ITT) effect is defined as
\begin{defn}{Intention-To-Treat (ITT) effect.}
   \begin{align*}
      ITT_Y = \mathbb{E} \left[ Y_i | Z_i = 1 \right] - \mathbb{E} \left[ Y_i | Z_i = 0 \right],
   \end{align*}
\end{defn}
which refers to the instrument's average effect. 
Based on the subgroups in \ref{defn:subgroups} and the proportions $\pi_{G_i}$, an IV setting can be formalized by the usual four assumptions (\ref{assump:exclusion}, \ref{assump:monotonicity}, \ref{assump:existence}, \ref{assump:unconfoundedness_iv}) following \cite{Angrist1996}: 

\begin{assump}{Exclusion restriction.}
   \label{assump:exclusion} 
   \begin{align*}
      Y_i(0) = Y_i(1), \text{ for } G_i \in \{AT, NT \}.
   \end{align*}
\end{assump}
\begin{assump}{Monotonicity.}
   \label{assump:monotonicity}
   \begin{align*}
      W_i(1) \geq W_i(0) \rightarrow \pi_D = 0.
   \end{align*}
\end{assump}
\begin{assump}{Existence of compliers.}
   \label{assump:existence}
   \begin{align*}
      P(W_i(0) < W_i(1)) > 0 \rightarrow \pi_C \ne 0.
   \end{align*}
\end{assump}
\begin{assump}{Unconfoundedness of the instrumental variable.}
   \label{assump:unconfoundedness_iv}
   \begin{align*}
      Z_i \perp (Y_i(0, 0), Y_i(0, 1), Y_i(1, 0), Y_i(1, 1), W_i(0), W_i(1)).
   \end{align*}
\end{assump}

If these assumptions \ref{assump:exclusion} - \ref{assump:unconfoundedness_iv} hold, the Complier Average Causal Effect (CACE) is identified as

\begin{defn}{Complier Average Causal Effect (CACE)}
   \label{defn:CACE}
   \begin{align*}
      \tau_{\text{CACE}} = \frac{\text{ITT}_{Y}}{\pi_C} = \frac{\mathbb{E}[Y_i \mid Z_i = 1] - \mathbb{E}[Y_i \mid Z_i = 0]}{\mathbb{E}[W_i \mid Z_i = 1] - \mathbb{E}[W_i \mid Z_i = 0]}.
   \end{align*}
\end{defn}

can be estimated from observational data. The numerator represents the average effect of the instrument, also referred to as the Intention-To-Treat (ITT) effect. 
The denominator represents the overall proportion of indiviudals that comply with the treatment assignment, also referred to as the proportion of compliers \cite{Angrist1996}.
CACE is also sometimes referred to as Local Average Treatment Effects (see \cite{AngristPischke2008}) and represents the estimate of the causal effect of the assignment to treatment on the principal outcome, $Y_i$, for the subpopulation of compliers \cite{ImbensRubin2015}.



Following \cite{bargagli-stoffi_heterogeneous_2022}, we consider the following conditional version of the CACE, 

\begin{defn}{Conditional CACE (cCACE).}
   \begin{align*}
      \tau_{\text{CACE}}(x) = \frac{\text{ITT}_{Y}(x)}{\pi_C(x)} = \frac{\mathbb{E}[Y_i \mid Z_i = 1, X_i = x] - \mathbb{E}[Y_i \mid Z_i = 0, X_i = x]}{\mathbb{E}[W_i \mid Z_i = 1, X_i = x] - \mathbb{E}[W_i \mid Z_i = 0, X_i = x]}.
   \end{align*}
\end{defn}

The cCACE is a straightforward extension of the CACE in \ref{defn:CACE} with \cite{bargagli-stoffi_heterogeneous_2022} providing theorems of a consistent and asymptotic normal conditional Two-Stage-Least-Square (2SLS) estimator which rely on the unconditional theorems of Wooldridge. 
The conditional theorems just need the additional assumption of sufficient number of observations for every leaf node in which the cCACE is estimated.








The conditional CACE can be estimated in a generic sub-sample (i.e., for each $\mathbf{X}_i \in \mathbb{X}_j$ , where $\mathbb{X}_j$ is a
generic node of the discovered tree, like a non-terminal node or a leaf) as:

\chapter{Sparsity-inducing Bayesian Causal Forest with
Instrumental Variable}

\section{Sparse Bayesian Causal Forest}


\section{Sparse BCF-IV}


\chapter{Simulation Study}


\section{Setup}

Let us generate the set of potential outcomes, potential treatments and covariates for each individual $i$ by  

\begin{align*}
    Z_i &\sim \text{Binom}(0.5), \\
    W_i(Z_i=1) &\sim \text{Binom}(\pi_{comp}), \\
    W_i(Z_i=0) &= 0, \\
    W^{obs}_i &= Z_i W_i(1) +  (1-Z_i) W_i(0), \\
    Y_i(Z_i=0) &\sim \mathcal{N}(0, 1), \\
    Y_i(Z_i=1) &= Y_i(0) + \mu(X_i) + W_i(1) \tau^{CACE}(X_i).
\end{align*}
Let $i=1, ..., N \text{ with } N = 1000$ be the number of observations. The covariates are binary as well as continuous variables generated by 
\begin{align*}
    X_{i,1}, ..., X_{i, P/2} &\sim \text{Binom}(0.5), \\
    X_{i, P/2+1}, ..., X_{i, P} &\sim \mathcal{N}(0, 1), 
\end{align*}
with the number of covariates $P \in \{10, 50, 100\}$ to illustrate different degrees of sparsity.
Moreover, we differ between different strengths of compliance rates with $\pi_{comp} \in \{0.5, 0.75\}$ such that we discriminate between a weak and a strong instrument scenario. With $\pi_{comp} << 1$, we guarantee an operation under imperfect compliance where individuals may not comply with their actual treatment assignment. More precisely, we use one-sided non-compliance here, due to $W_i(0)=0$, as the potential treatment receipt of individuals that are not assigend to treatment is always zero. The effect size differs by  $k \in \{1 ,2\}$ such that we can compare a small and a greater amount of heterogeneity between subgroups.
We generate the following simplistic scenario for the heterogeneous conditional CACE by conditioning $\tau^{CACE}(X_i)$ solely on the binary variables $X_{i, 1}$ and $X_{i, 2}$ using 
\begin{align*} 
    \tau^{CACE}(X_i) = \begin{cases*}
        k \text{, if } X_i \in l_1 = \{X_i: X_{i,1}=0, X_{i,2}=0 \}, \\
        -k \text{, if } X_i \in l_2 = \{X_i: X_{i,1}=1, X_{i, 2}=1 \}, \\
        0 \text{, else.}
    \end{cases*}
\end{align*}

The baseline effect $\mu(.)$ solely depends on continuous covariates and is either unconfounded by 
\begin{align*}
    \mu(X_i) = 3 + 1.5 \sin{\left(\pi X_{i, P/2 + 1}\right)} + 0.5\left(X_{i, P/2+2} - 0.5\right)^2 + 
    1.5\left(2-\abs{X_{i, P/2+3}}\right),
\end{align*}
or confounded by extending to
\begin{align*}
    \mu(X_i) = &3 + 1.5 \sin{\left(\pi X_{i, P/2 + 1}\right)} + 0.5\left(X_{i, P/2+2} - 0.5\right)^2 + 
    1.5\left(2-\abs{X_{i, P/2+3}}\right) + \\
    &1.5X_{i, P/2 + 4}\left(X_{i, 1} + 1\right),
\end{align*}
where the heterogeneity-inducing covariate $X_{i, 1}$ (used within the determination of $\tau^{CACE}(X_i)$) is also present within $\mu(.)$.


\section{Performance Criteria}

Performance criteria according to bargagli-stoffi: 

1. Average number of truly discovered heterogeneous subgroups corresponding to the nodes of the generated CART (proportion of correctly discovered subgroups);

2. Monte Carlo estimated bias for the heterogeneous subgroups:
\begin{equation}
\text{Bias}_m(I_\text{inf}) = \frac{1}{N_\text{inf}} \sum_{i=1}^{N_\text{inf}} \sum_{l=1}^{L} \left( \tau_{\text{cace},i}(\ell) - \hat{\tau}_{\text{cace},i}(\ell, \Pi_m, I_\text{inf}) \right),
\end{equation}
\begin{equation}
\text{Bias}(I_\text{inf}) = \frac{1}{M} \sum_{m=1}^{M} \text{Bias}_m(I_\text{inf}),
\end{equation}
where $\Pi_m$ is the partition selected in simulation $m$, $L$ is the number of subgroups with heterogeneous effects (i.e., two for the case of strong heterogeneity and four for the case of slight heterogeneity), and $N_\text{inf}$ is the number of observations in the inference sample.

3. Monte Carlo estimated MSE for the heterogeneous subgroups:
\begin{equation}
\text{MSE}_m(I_\text{inf}) = \frac{1}{N_\text{inf}} \sum_{i=1}^{N_\text{inf}} \sum_{l=1}^{L} \left( \tau_{\text{cace},i}(\ell) - \hat{\tau}_{\text{cace},i}(\ell, \Pi_m, I_\text{inf}) \right)^2,
\end{equation}
\begin{equation}
\text{MSE}(I_\text{inf}) = \frac{1}{M} \sum_{m=1}^{M} \text{MSE}_m(I_\text{inf});
\end{equation}

4. Monte Carlo coverage, computed as the average proportion of units for which the estimated 95\% confidence interval of the causal effect in the assigned leaf includes the true value, for the heterogeneous subgroups:
\begin{equation}
C_m(I_\text{inf}) = \frac{1}{N_\text{inf}} \sum_{i=1}^{N_\text{inf}} \sum_{l=1}^{L} \left( \tau_{\text{cace},i}(\ell) \in \hat{\text{CI}}_{95} \left( \hat{\tau}_{\text{cace},i}(\ell, \Pi_m, I_\text{inf}) \right) \right),
\end{equation}
\begin{equation}
C(I_\text{inf}) = \frac{1}{M} \sum_{m=1}^{M} C_m(I_\text{inf}).
\end{equation}




\section{Results}
  
 



\include{chapters/05_emp_appl}

\chapter{Discussion}


Further ideas \\

Open questions for further contributions. Use Random Forests (Causal Rule Ensemble) instead of single CART for subgroup discovery? 
More rigorous Bayesian estimation by replacing \texttt{ivreg} with \texttt{brms} (implementation of credible intervals, estimation error, counterparts to something like weak instrument tests, bayesian model averaging)? 



Theorems of conditional 2SLS \\ 

Moreover, \cite{Wooldridge2015} provide theorems for consistency and asymptotic normality for the unconditional TSLS estimator for the true population parameter $\tau_{CACE}$. \cite{bargagli-stoffi_heterogeneous_2022} show that those theorems can be transferred to the conditional TSLS estimator if there are sufficient number of observations for every subgroup in which the cCACE is estimated. 

\begin{thm}[Consistency and Asymptotic Normality of the Conditional 2SLS Estimator]
   Let Assumptions 1, 2, and 3 hold, i.e., \(E(Z^2_{i,X_j}) \neq 0\) (Assumption 1), \(E(Z_{i,X_j} \varepsilon_{i,X_j}) = 0\) (Assumption 2), and \(\pi_{C,X_j} \neq 0\) (Assumption 3). Then:
   \begin{enumerate}
       \item (Consistency) \( \hat{\tau}^{2SLS}_{X_j} - \tau_{X_j} \overset{p}{\to} 0 \) as \( N_{X_j} \to \infty \), where \( \overset{p}{\to} \) denotes convergence in probability, and \( N_{X_j} \) is the number of observations within the node \( X_j \).
       \item (Asymptotic Normality) If, in addition, \(E(Z^2_{i,X_j} \varepsilon^2_{i,X_j})\) is finite (Assumption 4), then:
       \[
       \sqrt{N_{X_j}} (\hat{\tau}^{2SLS}_{X_j} - \tau_{X_j}) \overset{d}{\to} \mathcal{N}(0, N_{X_j} \cdot \text{avar}(\hat{\tau}^{2SLS}_{X_j}))
       \]
       as \(N_{X_j} \to \infty\), where \( \overset{d}{\to} \) denotes convergence in distribution, \( \mathcal{N}(0, N_{X_j} \cdot \text{avar}(\hat{\tau}^{2SLS}_{X_j})) \) stands for the normal distribution, and \(\text{avar}(\hat{\tau}^{2SLS}_{X_j})\) is the asymptotic variance of the 2SLS estimator that can be approximated as in Chapter 15 of Wooldridge (2015).
   \end{enumerate}
\end{thm}











%\section{Appendices}
%\addcontentsline{toc}{section}{Appendices}



\Urlmuskip=0mu plus 1mu\relax
\bibliographystyle{agsm}
\bibliography{refs/BART_IV_paper}

\cleardoubleemptypage



\end{document}