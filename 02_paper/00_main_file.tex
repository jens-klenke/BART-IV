\documentclass[oneside, english, reqno, 11pt, headsepline, BCOR=8.5mm]{scrbook} %BCOR8.5mm, ohne DIV openany % exchange 'article' with 'scrbook' % oneside, english, reqno, 11pt, headsepline, BCOR=8.5mm
\usepackage[T1]{fontenc}
\usepackage[utf8]{inputenc}
\usepackage{lmodern}
%\setlength{\parskip}{\smallskipamount}
%\setlength{\parindent}{0pt}
%\usepackage{textcomp}
\usepackage{amsmath, amssymb, amsthm, amstext,mathrsfs, mathtools}
\newcommand\norm[1]{\left\lVert#1\right\rVert} % define adjustable norm
\usepackage{esint}
\usepackage{babel}
\usepackage{microtype}
\usepackage{relsize} % for \mathlarger
\usepackage[english=usenglishmax]{hyphsubst}	
\usepackage{booktabs}
\usepackage{scrlayer-scrpage}
\usepackage{graphicx}
%\usepackage{ngerman}
\usepackage[babel,german=quotes]{csquotes}
\usepackage{bibentry}
\usepackage{tabularx}	
\usepackage{hhline} % for double h line
\usepackage{commath}			% for \norm{...}
\usepackage{setspace, natbib}
\usepackage{bm}
\usepackage{threeparttable}
\usepackage{comment}\includecomment{notthisone}
\usepackage[colorinlistoftodos]{todonotes}
\usepackage{enumerate}
\usepackage{epstopdf} 
\usepackage{dsfont}
\usepackage{pstool}
\usepackage{subfig}
\usepackage{natbib}
\usepackage{hyperref}
\usepackage{scalerel} % for the \scaleto command
\usepackage{array, booktabs}
\usepackage{bbm}
\usepackage{setspace}
\usepackage{mathcomp}
\usepackage{multirow}
\usepackage[toc,page]{appendix}
\usepackage{lipsum}
\usepackage{titling} % for abstract
%\usepackage[top=3.5cm,bottom=4.5cm]{geometry}
\usepackage{verbatim} % to do a block comment
\usepackage{float} % figures tsay where they should be with "H"
\usepackage{dsfont} % for indicator function
\usepackage{fancyvrb} % for "\Verb" macro
\VerbatimFootnotes    % enable use of \Verb in footnotes

\renewcommand\thesubtable{(\alph{subtable})}

\DeclareOldFontCommand{\rm}{\normalfont\rmfamily}{\mathrm}
\DeclareOldFontCommand{\sf}{\normalfont\sffamily}{\mathsf}
\DeclareOldFontCommand{\tt}{\normalfont\ttfamily}{\mathtt}
\DeclareOldFontCommand{\bf}{\normalfont\bfseries}{\mathbf}
\DeclareOldFontCommand{\it}{\normalfont\itshape}{\mathit}

%\usepackage{algorithm}
%\usepackage{algcompatible}
%\usepackage{algorithmicx}
\usepackage[ruled]{algorithm2e}
%\usepackage[noend]{algpseudocode}

%\floatname{algorithm}{Algorithm}
%\renewcommand{\algorithmicrequire}{\textbf{Input:}}
%\renewcommand{\algorithmicensure}{\textbf{Output:}}

\usepackage{mathtools}
%\mathtoolsset{showonlyrefs} % activate this for no equation numbering

\makeatletter
\def\BState{\State\hskip-\ALG@thistlm}
\makeatother

%%This is style file ee.sty.
%%version 30/01/2002
%
\newcommand{\vnorm}[1]{\left|\!\left|#1\right|\!\right|}
%\newcommand{\lowexp}[1]{\raisebox{1.45mm}[-1.45mm]{\scriptsize{$#1$}}}
\newcommand{\lowexp}[2]{\raisebox{#2mm}[-#2mm]{\scriptsize{$#1$}}}
\newcommand{\newoperator}[3]{\newcommand*{#1}{\mathop{#2}#3}}
\newcommand{\renewoperator}[3]{\renewcommand*{#1}{\mathop{#2}#3}}
%
% symbols C,N,Q,R,Z for sets
\newcommand{\SC}{\mathbb{C}}
\newcommand{\SSS}{\mathbb{S}}
\newcommand{\SN}{\mathbb{N}}
\newcommand{\SQ}{\mathbb{Q}}
\newcommand{\SR}{\mathbb{R}}
\newcommand{\SZ}{\mathbb{Z}}
\newcommand{\SB}{\mathbb{B}}
%
% calligraphic capital letters
\newcommand{\calA}{\mathcal{A}}
\newcommand{\calB}{\mathcal{B}}
\newcommand{\calC}{\mathcal{C}}
\newcommand{\calD}{\mathcal{D}}
\newcommand{\calE}{\mathcal{E}}
\newcommand{\calF}{\mathcal{F}}
\newcommand{\calG}{\mathcal{G}}
\newcommand{\calH}{\mathcal{H}}
\newcommand{\calI}{\mathcal{I}}
\newcommand{\calJ}{\mathcal{J}}
\newcommand{\calK}{\mathcal{K}}
\newcommand{\calL}{\mathcal{L}}
\newcommand{\calM}{\mathcal{M}}
\newcommand{\calN}{\mathcal{N}}
\newcommand{\calO}{\mathcal{O}}
\newcommand{\calP}{\mathcal{P}}
\newcommand{\calQ}{\mathcal{Q}}
\newcommand{\calR}{\mathcal{R}}
\newcommand{\calS}{\mathcal{S}}
\newcommand{\calT}{\mathcal{T}}
\newcommand{\calU}{\mathcal{U}}
\newcommand{\calV}{\mathcal{V}}
\newcommand{\calW}{\mathcal{W}}
\newcommand{\calX}{\mathcal{X}}
\newcommand{\calY}{\mathcal{Y}}
\newcommand{\calZ}{\mathcal{Z}}
%
% bold lowercase and capital letters for vectors (v) and matrices (m)
\newcommand{\mA}{\bm A}
\newcommand{\va}{\bm a}
\newcommand{\mB}{\bm B}
\newcommand{\vb}{\bm b}
\newcommand{\mC}{\bm C}
\newcommand{\vc}{\bm c}
\newcommand{\mD}{\bm D}
\newcommand{\vd}{\bm d}
\newcommand{\mE}{\bm E}
\newcommand{\ve}{\bm e}
\newcommand{\mF}{\bm F}
\newcommand{\vf}{\bm f}
\newcommand{\mG}{\bm G}
\newcommand{\vg}{\bm g}
\newcommand{\mH}{\bm H}
\newcommand{\vh}{\bm h}
\newcommand{\mI}{\bm I}
\newcommand{\vi}{\bm i}
\newcommand{\mJ}{\bm J}
\newcommand{\vj}{\bm j}
\newcommand{\mK}{\bm K}
\newcommand{\vk}{\bm k}
\newcommand{\mL}{\bm L}
\newcommand{\vl}{\bm l}
\newcommand{\mM}{\bm M}
\newcommand{\vm}{\bm m}
\newcommand{\mN}{\bm N}
\newcommand{\vn}{\bm n}
\newcommand{\mO}{\bm O}
\newcommand{\vo}{\bm o}
\newcommand{\mP}{\bm P}
\newcommand{\vp}{\bm p}
\newcommand{\mQ}{\bm Q}
\newcommand{\vq}{\bm q}
\newcommand{\mR}{\bm R}
\newcommand{\vr}{\bm r}
\newcommand{\mS}{\bm S}
\newcommand{\vs}{\bm s}
\newcommand{\mT}{\bm T}
\newcommand{\vt}{\bm t}
\newcommand{\mU}{\bm U}
\newcommand{\vu}{\bm u}
\newcommand{\mV}{\bm V}
\newcommand{\vv}{\bm v}
\newcommand{\mW}{\bm W}
\newcommand{\vw}{\bm w}
\newcommand{\mX}{\bm X}
\newcommand{\vx}{\bm x}
\newcommand{\mY}{\bm Y}
\newcommand{\vy}{\bm y}
\newcommand{\mZ}{\bm Z}
\newcommand{\vz}{\bm z}
\newcommand{\vzero}{\bm 0}
\newcommand{\vone}{\bm 1}
%
% bold Greek lowercase letters for vectors (v)
\newcommand{\valpha}{\bm \alpha}
\newcommand{\vbeta}{\bm \beta}
\newcommand{\vgamma}{\bm \gamma}
\newcommand{\vdelta}{\bm \delta}
\newcommand{\vepsi}{\bm \epsi}
\newcommand{\vvarepsilon}{\bm \varepsilon}
\newcommand{\vzeta}{\bm \zeta}
\newcommand{\veta}{\bm \eta}
\newcommand{\vtheta}{\bm \theta}
\newcommand{\viota}{\bm \iota}
\newcommand{\vkappa}{\bm \kappa}
\newcommand{\vlambda}{\bm \lambda}
\newcommand{\vmu}{\bm \mu}
\newcommand{\vnu}{\bm \nu}
\newcommand{\vxi}{\bm \xi}
\newcommand{\vpi}{\bm \pi}
\newcommand{\vrho}{\bm \rho}
\newcommand{\vsigma}{\bm \sigma}
\newcommand{\vtau}{\bm \tau}
\newcommand{\vupsilon}{\bm \upsilon}
\newcommand{\vphi}{\bm \phi}
\newcommand{\vchi}{\bm \chi}
\newcommand{\vpsi}{\bm \psi}
\newcommand{\vomega}{\bm \omega}
%
% bold Greek capital letters for matrices (m)
\newcommand{\mGamma}{\bm \varGamma}
\newcommand{\mDelta}{\bm \varDelta}
\newcommand{\mTheta}{\bm \varTheta}
\newcommand{\mLambda}{\bm \varLambda}
\newcommand{\mXi}{\bm \varXi}
\newcommand{\mXib}{\bm \Xi}
\newcommand{\mPi}{\bm \varPi}
\newcommand{\mSigma}{\bm \varSigma}
\newcommand{\mUpsilon}{\bm \varUpsilon}
\newcommand{\mPhi}{\bm \varPhi}
\newcommand{\mPsi}{\bm \varPsi}
\newcommand{\mOmega}{\bm \varOmega}
%
% roman letters in mathematics
\newcommand{\rb}{\ensuremath{\mathrm{b}}}
\newcommand{\rB}{\ensuremath{\mathrm{B}}}
\newcommand{\rC}{\ensuremath{\mathrm{C}}}
\newcommand{\rD}{\ensuremath{\mathrm{D}}}
\newcommand{\rf}{\ensuremath{\mathrm{f}}}
\newcommand{\rF}{\ensuremath{\mathrm{F}}}
\newcommand{\rH}{\ensuremath{\mathrm{H}}}
\newcommand{\rL}{\ensuremath{\mathrm{L}}}
\newcommand{\rN}{\ensuremath{\mathrm{N}}}
\newcommand{\rt}{\ensuremath{\mathrm{t}}}
\newcommand{\rU}{\ensuremath{\mathrm{U}}}
\newcommand{\rGam}{\ensuremath{\mathrm{Gam}}}
\newcommand{\rBeta}{\ensuremath{\mathrm{Beta}}}
%
\newcommand{\Bin}{\ensuremath{\mathrm{Bin}}}
\newcommand{\eu}{\ensuremath{\mathrm{e}}}
\newcommand{\iu}{\ensuremath{\mathrm{i}}}
\newcommand{\LN}{\ensuremath{\mathrm{LN}}}
\newcommand{\IN}{\ensuremath{\mathrm{IN}}}

\newcommand{\Poi}{\ensuremath{\mathrm{Poi}}}
%
\newcommand{\ped}[1]{\ensuremath{_\mathrm{#1}}} %pedex
\newcommand{\ap}[1]{\ensuremath{^\mathrm{#1}}} %apex
\renewoperator{\Re}{\mathrm{Re}}{\nolimits}
\renewoperator{\Im}{\mathrm{Im}}{\nolimits}
%
% letters for (partial) differentiation
%\newcommand{\rd}{\ensuremath{\mathrm{d}}}
\makeatletter
\newcommand{\rd}{\@ifnextchar^{\DIfF}{\DIfF^{}}}
\def\DIfF^#1{%
   \mathop{\mathrm{\mathstrut d}}%
   \nolimits^{#1}\gobblespace}
\def\gobblespace{\futurelet\diffarg\opspace}
\def\opspace{%
   \let\DiffSpace\!%
   \ifx\diffarg(%
   \let\DiffSpace\relax
   \else
   \ifx\diffarg[%
   \let\DiffSpace\relax
   \else
   \ifx\diffarg\{%
   \let\DiffSpace\relax
   \fi\fi\fi\DiffSpace}
\newcommand{\deriv}[3][]{\frac{\rd^{#1}#2}{\rd #3^{#1}}}
\newcommand{\pderiv}[3][]{\frac{\partial^{#1}#2}{\partial #3^{#1}}}
%
% operatornames
\newcommand{\Avar}{\operatorname{Avar}}
\newcommand{\bias}{\operatorname{bias}}
\newcommand{\col}{\operatorname{col}}
\newcommand{\corr}{\operatorname{corr}}
\newcommand{\Corr}{\operatorname{Corr}}
\newcommand{\cov}{\operatorname{cov}}
\newcommand{\Cov}{\operatorname{Cov}}
\newcommand{\dg}{\operatorname{dg}}
\newcommand{\diag}{\operatorname{diag}}
\newcommand{\E}{\operatorname{E}}
\newcommand{\etr}{\operatorname{etr}}
\newoperator{\ip}{\mathrm{int}}{\nolimits}
\newcommand{\kur}{\operatorname{kur}}
%\newcommand{\median}{\operatorname{med}}
\newcommand{\MSE}{\operatorname{MSE}}
\newcommand{\plim}{\operatorname{plim}}
%\newcommand{\plim}{\DeclareMathOperator{plim}}
\newcommand{\cond}{\rightarrow_{\mathrm{d}}}
\newcommand{\conp}{\rightarrow_{\mathrm{p}}}
\newcommand{\rk}{\operatorname{rk}}
\newcommand{\sgn}{\operatorname{sgn}}
\newcommand{\spur}{\operatorname{spur}}
\newcommand{\tr}{\operatorname{tr}}
\newcommand{\var}{\operatorname{Var}}
\newcommand{\Var}{\operatorname{Var}}
\renewcommand{\vec}{\operatorname{vec}}
\newcommand{\vech}{\operatorname{vech}}
\DeclareMathOperator*{\argmin}{arg\,min}
\DeclareMathOperator*{\argmax}{arg\,max}
%
% other definitions
\newcommand{\distr}{\sim}
\newcommand{\adistr}{\stackrel{a}{\distr}}
\newcommand{\diff}{\Delta}
\newcommand{\fordiff}{\bigtriangleup}
\newcommand{\fdiff}{\diff_{\rf}}
\newcommand{\bdiff}{\diff_{\rb}}
%
%\mathchardef\varepsilon="010F
%\mathchardef\epsilon="0122
%\mathchardef\eps="010F
\newcommand{\eps}{\epsilon}
\newcommand{\epsi}{\varepsilon}
%
\newcommand{\longto}{\longrightarrow}
\newcommand{\pto}{\stackrel{p}{\longrightarrow}}
\newcommand{\dto}{\stackrel{d}{\longrightarrow}}
\newcommand{\wto}{\stackrel{w}{\longrightarrow}}
%
\newcommand{\Infmat}{\bm\calI}
\newcommand{\Hesmat}{\bm\calH}
\newcommand{\bcdot}{\raisebox{1pt}{\textbf{\large .}}}
\newcommand{\interior}[1]{\overset{\circ}{#1}}
%
\newcommand{\vones}{\bm\imath}
\newcommand{\vzeros}{\boldsymbol{0}}
\newcommand{\mZeros}{\mathbf{O}}
%
% additional commands
\newcommand{\EE}[1]{\E\left(#1\right)}
\renewcommand{\Pr}{\ensuremath{\mathrm{P}}}
\newcommand{\prob}[1]{\Pr\left(#1\right)}
\newcommand{\Prob}[1]{\Pr\left(#1\right)}
\newcommand{\Spur}[1]{\spur\left[#1\right]}
\newcommand{\vvarphi}{\bm \varphi}
%
 


%\usepackage{cite} %Literaturverwaltung 
%\usepackage{setspace} % Zeilenabstand aendern

%\setcounter{tocdepth}{3}

%\bibliographystyle{plain}
\pagestyle{scrheadings}%{plain}%
\makeatletter

\newtheoremstyle{Definition}
  {0.2cm}                   %Space above
  {0.2cm}                   %Space below
  {\normalfont}           %Body font
  {}                      %Indent amount (empty = no indent,
                          %\parindent = para indent)
  {\bfseries}  						%Thm head font
  {.}                     %Punctuation after thm head
  { }              				%Space after thm head: " " = normal interword
                          %space; \newline = linebreak
  {}
                          %Thm head spec (can be left empty, meaning
                          %`normal')
                          
\newtheoremstyle{Theorem}
  {0.2cm}                   %Space above
  {0.2cm}                   %Space below
  {\itshape}           		%Body font
  {}                      %Indent amount (empty = no indent,
                          %\parindent = para indent)
  {\bfseries}  						%Thm head font
  {.}                     %Punctuation after thm head
  { }              				%Space after thm head: " " = normal interword
                          %space; \newline = linebreak
  {}
                          %Thm head spec (can be left empty, meaning
                          %`normal')


%%%%%%%%%%%%%%%%%%%%%%%%%%%%%% Textclass specific LaTeX commands.
\numberwithin{equation}{section}
\numberwithin{figure}{section}


\theoremstyle{Theorem}
	\newtheorem{cor}{Corollary}[chapter]
	\newtheorem{prop}{Proposition}[chapter]
	\newtheorem{lem}{Lemma}[chapter]
	\newtheorem{thm}{Theorem}[chapter]
	\newtheorem{thmA}{Theorem A}[chapter]
	\newtheorem{lemA}{Lemma A}[chapter]

\theoremstyle{Definition}
	\newtheorem{example}{Example}[chapter]
	\newtheorem{rem}{Remark}[chapter]
	\newtheorem{defn}{Definition}[chapter]
    \newtheorem{assump}{Assumption}[chapter]
	
\newcommand\Chapter[2]{\chapter
  [#1\hfil\hbox{}\protect\linebreak{\itshape#2}]%
  {#1\\[2ex]\itshape#2}%
}

\renewcommand\topfraction{0.85}
\renewcommand\bottomfraction{0.85}
\renewcommand\textfraction{0.1}
\renewcommand\floatpagefraction{0.85}

\setlength{\footskip}{2cm}
\setlength{\parindent}{2em}
\setlength{\parskip}{.5em}

\newcommand{\Levy}{L\'{e}vy }


\allowdisplaybreaks

\raggedbottom

\makeatother

%%%%%%%%%%%%%%%%%%%%%%%%%%%%%%%%%%%%%%%%%%%%%%%%%%%%%%%%%%%
% just some information for \maketitle
\title{Shrinkage Bayesian Causal Forest with Instrumental Variable}
\subtitle{Here should be a fancy subtitle}
\author
{Jens Klenke$^{1,\barwedge}$, Lennard Ma{\ss}mann$^{1,2,\ast}$ \\
\\
\normalsize{$^{1}$Chair of Econometrics, University of Duisburg-Essen}\\
\normalsize{$^{2}$Ruhr Graduate School in Economics}\\
\\
\normalsize{$^\ast$E-mail:  lennard.massmann@uni-due.de}\\
\normalsize{$^\barwedge$E-mail:  jens.klenke@vwl.uni-due.de}\\
}
\date{\today}
%\publishers{Aperture Science Enrichment Center}

%%%%%%%%%%%%%%%%%%%%%%%%%%%%%%%%%%%%%%%%%%%%%%%%%%%%%%%%%%%

%%%%%%%% to introduce an abstract %%%%%%%%%
%   Reduce the margin of the summary:
\def\changemargin#1#2{\list{}{\rightmargin#2\leftmargin#1}\item[]}
\let\endchangemargin=\endlist 

%   Generate the environment for the abstract:
\newcommand\summaryname{Abstract}
\newenvironment{Abstract}%
    {\small\begin{center}%
    \bfseries{\summaryname} \end{center}}
%%%%%%%% to introduce an abstract %%%%%%%%%



\newcommand{\Rbb}{\mathbb{R}}
\newcommand{\Nbb}{\mathbb{N}}
\newcommand{\Zbb}{\mathbb{Z}}
\newcommand{\Qbb}{\mathbb{Q}}
\newcommand{\Ebb}{\mathbb{E}}
\newcommand{\ind}{\perp\!\!\!\!\perp} % statistical independence sign


%%%%%%%% todo comments %%%%%%%%%

\newcommand{\todoComment}[2][]{\todo[linecolor=red!20,backgroundcolor=red!20,bordercolor=red!20, size=\tiny,#1]{#2}}



\begin{document}

%\onehalfspacing % 1.5-facher Zeilenabstand
\frontmatter
%\input{title}
\maketitle

%\listoftodos

\cleardoublepage


\begin{Abstract}
\begin{changemargin}{1cm}{1cm}
  This paper focuses on improving the estimation of heterogeneous treatment effects in observational studies under sparsity and conditions of imperfect compliance using instrumental variables (IV).
  Traditional IV methods, such as two-stage least squares (2SLS), often impose linearity assumptions that may not hold in complex empirical settings.
  To address these limitations, the Bayesian Instrumental Variable Causal Forest (BCF-IV) framework has been developed to estimate the conditional Complier Average Causal Effect (CACE) non-parametrically while retaining interpretability.
  BCF-IV, based on the Bayesian Additive Regression Trees (BART) algorithm, identifies treatment effect heterogeneity and estimates conditional CACE using 2SLS leafwise. 
  This research contributes in two significant ways by proposing the Shrinkage Bayesian Instrumental Variable Causal Forest (SBCF-IV) algorithm. 
  First, the paper adopts the SoftBART algorithm, which integrates shrinkage priors to enable more precise treatment effect estimation in the presence of sparse data.
  Second, the paper refines the discovery of heterogeneous subgroups by incorporating posterior splitting probabilities into the decision-making process.
  These probabilities are used to scale the cost function during subgroup construction, leading to more accurate identification of meaningful subgroups.
  The approach of SBCF-IV enhances the ability to manage sparse data and improves the detection of variables that drive treatment effect heterogeneity. A simulation study suggests improved adaptability and interpretability in estimating conditional CACE, particularly in scenarios with sparsity, confounding and nonlinearity.
\end{changemargin}
\end{Abstract}

\cleardoublepage

\pagenumbering{Roman}

% \begin{center}
% \textbf{Frequently used symbols and abbreviations:}
% \end{center}
% \begin{eqnarray*}
% \mathcal{S}^{tr} &:& \textup{training sample (for tree-growing process)} \\
% \mathcal{S}^{est} &:& \textup{estimation sample (for leaf-wise estimation)} \\
% \mathcal{S}^{te} &:& \textup{test sample} \\
% \Pi &:& \textup{tree/partitioning of feature space } \mathbb{X}\\
% MSE &:& \textup{mean squared error} \\
% \mu(x; \mathcal{S}, \Pi) &:& \textup{conditional mean function} \\
% \tau(x; \mathcal{S}, \Pi) &:& \textup{conditional average treatment effect} \\
% \tau_i &:& \textup{unit-level treatment effect} \\
% X_i &:& \textup{individual $i$'s feature matrix of dimension } N \times K \\
% D_i &:& \textup{individual $i$'s binary treatment indicator, $D_i \in (0,1)$} \\
% Y_i(D_i) &:& \textup{individual $i$'s potential outcome}\\
% Y_i^{obs} &:& \textup{individual $i$'s observed potential outcome}\\
% \mathbb{E} &:& \textup{expectation operator}\\
% Pr(.) &:& \textup{probability}\\
% Pr(.|.) &:& \textup{conditional probability}\\
% \hspace{1cm} \\
% \textup{to be continued ...}
% \end{eqnarray*}


\begin{comment}
\mathbb{N} &:& \{1,2,\ldots\}\\
\mathbb{R} &:& \textup{set of real numbers}\\
%\mathbb{R}_+ &:& (0,\infty)\\
\mathbb{C} &:& \textup{set of complex numbers}\\
\mathrm{i} &:& \textup{imaginary unit, }  \mathrm{i}^2=-1\\
\mathrm{Re}(z) &:& \textup{real part } a \textup{ of } z=a+b\mathrm{i}\in\mathbb{C}\\
\arg(z) &:& \textup{the argument of a complex number } z\\
A^{\mathrm{T}} &:& \textup{transpose of a matrix } A\in\mathbb{R}^{m\times n}\\
\left\langle x,y\right\rangle &:& x^{\mathrm{T}}y, \textup{ for } x,y\in\mathbb{R}^n\\
\mathds{1}_A &:& \textup{indicator function } \mathds{1}_A(x)=1, \textup{ if } x\in A;\ \mathds{1}_A(x)=0, \textup{ if } x\notin A\\
\Gamma(x) &:& \textup{gamma function for } x>0\\
J_{\lambda}(x) &:& \textup{Bessel function of the first kind}\\
Y_{\lambda}(x) &:& \textup{Bessel function of the second kind}\\
K_{\lambda}(x) &:& \textup{modified Bessel function of the second kind}\\
f(x)\sim g(x) &:& \lim_{x\to a}\frac{f(x)}{g(x)}=1, a\in\mathbb{R}\cup\{\pm\infty\}\\
f(x)=\mathcal{O}( g(x)) &:& \limsup_{x\to a}\frac{|f(x)|}{|g(x)|}<\infty, a\in\mathbb{R}\cup\{\pm\infty\}\\
X\sim \mu &:& X \textup{ has distribution } \mu\\
\stackrel{\mathcal{L}}{=} &:& \textup{equality in distribution}\\
\stackrel{\mathcal{L}}{\rightarrow} &:& \textup{convergence in distribution}\\
X_n\stackrel{\mathcal{L}}{\sim}Y_n &:& X_n\stackrel{\mathcal{L}}{\rightarrow}X,Y_n\stackrel{\mathcal{L}}{\rightarrow}X\\
X_n=o_P(1) &:& X_n\to0 \textup{ in probability for } n\to\infty\\
\textup{a.s.} &:& \textup{almost surely}\\
\textup{i.i.d.} &:& \textup{independent, identically distributed}\\
N(\mu,\sigma^2) &:& \textup{normal distribution with mean } \mu \textup{ and variance } \sigma^2\\
t(\nu,\mu,\sigma^2) &:& \textup{Student } t \textup{ distribution with degree of freedom } \nu, \textup{ location } \mu \textup{ and scale } \sigma\\
t(\nu,\mu,\sigma^2,\beta) &:& \textup{Skew Student } t \textup{ distribution with skewness parameter } \beta\\
R\Gamma(\alpha,\beta) &:& \textup{inverse gamma distribution with shape parameter } \alpha \textup{ and scale } \beta\\
Poi(\lambda) &:& \textup{Poisson distribution with intensity parameter } \lambda\\
Exp(\lambda) &:& \textup{Exponential distribution with rate } \lambda\\
\mathcal{U}_{[a,b]} &:& \textup{Uniform distribution on } [a,b] 
\end{comment}


%\newpage{}

%\thispagestyle{empty}
\tableofcontents
%\newpage{}


\mainmatter
%\pagenumbering{arabic}
\microtypesetup{activate=true}


\chapter{Introduction}

\cite{hahn_bayesian_2020} 

\section{Literature}


\begin{itemize}
    \item Using machine learning to infer heterogeneous effects in observational studies often focuses on CATE estimation under regular assignment mechanisms 
    %($athey_generalized_2019$)
    \item Focus in this work: Methods to discover heterogeneous effects in the presence of imperfect compliance
    \item Methods
    \begin{itemize}
        \item tree-based 
        %($bargagli_stoffi_causal_2020$)
        %<!-- (perform worse compared to ensembles) -->
        \item ensemble-of-trees 
        %($@athey_generalized_2019$)
        %<!-- (rely on large samples, hart to interpret, Hahn 2019, Wendling 2018) -->
        \item deep-learning-based methods 
        %($@hartford_counterfactual_2016$)
        %<!-- (computationally intensive, hyper-parameter-sensitive, intolerant against unmeasured variation, hart to interpret) -->
    \end{itemize}
    \item BCF-IV: Discovers and estimates HTE in an interpretable way %($@bargagli-stoffi_heterogeneous_2020$)
    \begin{itemize}
        \item BCF: BART-based semi-parametric Bayesian regression model, able to estimate HTE in regular assignment mechanisms, even with strong confounding %($@hahn_bayesian_2020$)
        \item Use BCF to estimate $\hat\tau_C(x)$ and  $\widehat{ITT}_{Y}(x)$ such that the conditional Complier Average Causal Effect $\hat\tau^{cace}(x) = \frac{\widehat{ITT}_{Y}(x)}{\hat\tau_C(x)}$
    \end{itemize}
\end{itemize}



\section{Contribution}

\begin{itemize}
    \item - BART benefits %(@linero_softbart_2022):
    \begin{itemize}
        \item good performance in high-noise settings 
        \item shrinkage to/emphasize on low-order interactions
        \item established software implementations (`BayesTree`, `bartMachine`, `dbarts`)
    \end{itemize}  
    \item BART shortcomings %(@linero_softbart_2022, @hahn_bayesian_2020):
    \begin{itemize}
        \item non-smooth predictions as BART prior produces stepwise-continuous functions
        \item BART prior is overconfident in regions with weak common support
    \end{itemize}
    \item Research proposal: Rewrite the BCF-IV model with SoftBART instead of BART prior to account for sparsity
\end{itemize}


\chapter{Potential Outcomes and IV}



\section{Potential Outcomes and ITT}

\begin{itemize}
 \item $Y_i$: outcome variable; $W_i$: treatment variable; $Z_i$: instrumental variable
 \item $\mathbb{X}$: $N \times P$ matrix of control variables
 \item $G_i$: sub-populations of units
 \begin{itemize}
    \item $G_i=C: W_i(Z_i = 0) = 0, Wi(Zi = 1) = 1$
    \item $G_i=D: W_i(Z_i = 0) = 1, W_i(Z_i = 1) = 0$
    \item $G_i=AT: W_i(Z_i = 0) = 1, W_i(Z_i = 1) = 1$
    \item $G_i=NT: W_i(Z_i = 0) = 0, W_i(Z_i = 1) = 0$
 \end{itemize}
\end{itemize} 

Given the Stable Unit Treatment Value Assumption (SUTVA), one can postulate the existence of potential outcomes $Y_i(W_i)$ such that $Y_i^{obs} = Y_i(1)W_i + Y_i(0)(1-W_i)$.

One can directly get from the data the effect of the assignment to treatment: 
$$
\begin{aligned}
ITT_Y &= \mathbb{E}[Y_i|Z_i=1] - \mathbb{E}[Y_i|Z_i=0] \\
&= \pi_C ITT_{Y,C} + \pi_D ITT_{Y,D} + \pi_{AT} ITT_{Y,AT} + \pi_{NT} ITT_{Y,NT}.
\end{aligned}
$$


\section{IV assumptions}

Assumptions to infer Complier Average Causal Effect (CACE),  

$$
\tau^{cace} = ITT_{Y,C} = \frac{\mathbb{E}[Y_i|Z_i=1]-\mathbb{E}[Y_i|Z_i=0]}{\mathbb{E}[W_i|Z_i=1]-\mathbb{E}[W_i|Z_i=0]} = \frac{ITT_Y}{\pi_C}, 
$$ 

and its conditional version (cCACE),

$$
\tau^{cace}(x) = \frac{\mathbb{E}[Y_i|Z_i=1, \mathbb{X}_i=x]-\mathbb{E}[Y_i|Z_i=0, \mathbb{X}_i=x]}{\mathbb{E}[W_i|Z_i=1, \mathbb{X}_i=x]-\mathbb{E}[W_i|Z_i=0, \mathbb{X}_i=x]} = \frac{ITT_Y(x)}{\pi_C(x)}. 
$$ 

\hspace{0.2cm}\\
1. exclusion restriction: $Y_i(0) = Y_i(1), \text{ for } G_i \in \{AT, NT \}.$ \\
2. monotonicity: $W_i(1) \geq W_i(0) \rightarrow \pi_D = 0.$ \\
3. existence of compliers: $P(W_i(0) < W_i(1)) > 0 \rightarrow \pi_C \ne 0.$ \\
4. unconfoundedness of the instrument: $Z_i \perp (Y_i(0, 0), Y_i(0, 1), Y_i(1, 0), Y_i(1, 1), W_i(0), W_i(1)).$


\section{Estimation}

The conditional CACE can be estimated in a generic sub-sample (i.e., for each $\mathbf{X}_i \in \mathbb{X}_j$ , where $\mathbb{X}_j$ is a
generic node of the discovered tree, like a non-terminal node or a leaf) as:

\chapter{Bayesian Causal Forests and
Instrumental Variables}

This paper proposes an algorithm for the estimation of cCACE for sparse data scenarios. 
More precisely, we propose an extension of the BCF-IV algorithm in \cite{bargagli-stoffi_heterogeneous_2022} 
to handle scenarios with many irrelevant covariates in the dataset.
Section \ref{sec:BCF-IV} describes the original BCF-IV algorithm while section
\ref{sec:SBCF-IV} explains the proposed extension based on the Shrinkage Bayesian Causal Forest. As pointed out in the literature review of this paper, current ensemble methods that operate under an irregular assignment mechanism with imperfect compliance face some difficulties. Algorithms like Deep IV \cite{hartford_deep_2017} and the Generalized Random Forest \cite{athey_generalized_2019} provide precise cCACE estimates but are rather uninformative about relevant covariates, or subsets of possibly many covariates, that drive heterogeneity in cCACE. Tree-based methods like the Causal Tree with IV \cite{bargagli_stoffi_causal_2020} propose to estimate treatment effects under imperfect compliance and the existence of a suitable IV while retainig interpretability. However, although the single tree structure enables interpretability it also lacks of stability and replicability. \cite{bargagli-stoffi_heterogeneous_2022} argue to overcome those shortcomings using the BCF-IV algorithm as outlined in the following Section \ref{sec:BCF-IV}.   


\section{BCF-IV}
\label{sec:BCF-IV}

The main steps of the BCF-IV algorithm are outlined in Algorithm \ref{algo:BCF-IV}. Details of these three steps of honest sample splitting, discovery of treatment effect heterogeneity and inference of treatment effects are discussed in Sections \ref{honest_splitting}, \ref{discovery} and \ref{inference}.

\subsection{Honest sample splitting}
\label{honest_splitting}
The first step concerns honest sample splitting. This step enables a data-driven discovery of heterogeneous subgroups such that there is no need to specify those subgroups beforehand. Defining subgroups before estimating treatment effects based on relevant data of the studied population is a challenging task. It requires deep knowledge about the intricacies of the treatment effect at hand and may be prone to overlook relevant subgroups. Honest sample splitting as proposed in \cite{Athey_imbens_2016} is a remedy for those issues by making distinctions between model selection and treatment effect inference.    

\begin{algorithm}[H]
    \footnotesize
    \DontPrintSemicolon
    \SetAlgoLined
    \LinesNotNumbered
    \SetKwInOut{Input}{Input}\SetKwInOut{Output}{Output}
    %\SetKwBlock{BARTBMA}{BART-BMA(Y,x, hyperparams)}{} 
    \Input{$N$ units $i$ $(X_i, Z_i, W_i, Y_i)$, with feature vector $X_i$, treatment assignment (instrumental variable) $Z_i$, treatment receipt $W_i$, observed response $Y_i$} 
    \Output{A tree structure discovering the heterogeneity in the causal effects and estimates of the Complier Average Causal Effects (CACE) within its leaves.}
    \BlankLine
    \textbf{1. The Honest Splitting Step:} \\
    \begin{itemize}
        \item Randomly split the total sample into a discovery subsample ($\mathcal{I}_{\text{dis}}$) and an inference subsample ($\mathcal{I}_{\text{inf}}$).
    \end{itemize}
    \BlankLine
    \textbf{2. The Discovery Step} (performed on $I_{\text{dis}}$): \\
    Estimation of the Conditional CACE:
    \begin{itemize}
        \item (a) Estimate the conditional Intention-To-Treat via BCF: $\widehat{\text{ITT}}(x)$.
        \item (b) Estimate the conditional proportion of compliers via BART: $\widehat{\pi}_C(x)$.
        \item (c) Estimate the conditional CACE, $\widehat{\tau}_{\text{CACE}}(x)$, using the estimated values from (a) and (b).
    \end{itemize}
    Heterogeneous subpopulations discovery:
    \begin{itemize}
        \item (d) Discover the heterogeneous effects by fitting a decision tree using the data $(\widehat{\tau}_{\text{CACE}}(x), X_i)$.
    \end{itemize}
    \BlankLine
    \textbf{3. The Inference Step} (performed on $I_{\text{inf}}$): \\
    \begin{itemize}
        \item (a) Estimate the $\widehat{\tau}_{\text{CACE}}(x)$ for all discovered subpopulations (i.e., nodes and leaves) in the tree discovered in Step 3(d).
        \item (b) Perform multiple hypothesis tests and adjust p-values to control for the familywise error rate or, less stringently, the false discovery rate. 
        \item (c) Run weak-instrument tests within every node and discard nodes where weak-instrument issues are detected.
    \end{itemize}
    \caption{Bayesian Causal Forest with Instrumental Variable (BCF-IV)}
    \label{algo:BCF-IV}
\end{algorithm}


\subsection{Discovery of heterogeneous subgroups}
\label{discovery}

To estimate the conditional CACE given in Definition \ref{defn:cCACE}, we need some functional expression for the conditional expected value for the outcome, $Y_i$, as well as for the treatment indicator, $W_i$.  
The Bayesian Causal Forest (BCF) algorithm is proposed in \cite{hahn_bayesian_2020} to use the Bayesian Additive Regression Trees (BART) algorithm \cite{chipman_bart_2010} to estimate the CATE of Definition \ref{defn:CATE} in a regular assignment mechanism. BART is related to the CART algorithm of \cite{Breiman1984} which constructs binary trees by recursively partitioning the covariate space to produce accurate predictions. 
BART rests on a complete Bayesian probability model by using different regularizing prior distributions such that the overall model fit dominates fits of single trees. Distinct prior distributions are used for the complexity of the tree structure, data shrinkage within the nodes and the variance of the error term. The same idea is now transferred to the setup of an irregular assignment mechanism.
We start with modeling the numerator of Definition \ref{defn:cCACE} and restrict our dataset to observations within $\mathcal{I}_{disc}$.
Let the outcome variable $Y_i$ be modelled semi-parametrically as 
\begin{align*}
    Y_i = f(Z_i, X_i) + \varepsilon_i 
\end{align*}
with $\varepsilon_i \sim \left(0, \sigma_{\varepsilon}^2\right)$. The conditional expected value of the outcome be defined similar to \cite{hahn_bayesian_2020} as 

\begin{align*}
    \mathbb{E}\left[Y_i | Z_i = z, X_i=x \right] = \mu\left(\pi(x), x\right) + ITT_y(x)z. 
\end{align*}
Therefore, the conditional expectation is mainly governed by two additive functions. 
The first additive term $\mu\left(\pi(x), x\right)$ incorporates the IV's propensity score $\pi(x)=\mathbb{E}[Z_i=1 | X_i=x]$ and accounts for the direct, treatment-independent influence of the control variables on the outcome variable. 
The usage of the propensity score $\pi(x)$ prevents targeted selection and regularization-induced confounding. 
The second additive component $ITT_y(x)$ accounts for the direct, possibly heterogeneous, intention-to-treat effect. 
While the first term follows the standard prior specifications as in \cite{chipman_bart_2010}, the second term uses alternative tree depth penalty parameters $(\eta = 3, \beta=0.25)$ that encourage rather simplistic trees \cite{hahn_bayesian_2020}.
Consequently, we can get estimates of $\widehat{ITT}_y(x)$ as required in the discovery step 2(a) of Algorithm \ref{algo:BCF-IV}.
Analogously, we can model the denominator of Definition \ref{defn:cCACE} by still restricting our dataset to observations within $\mathcal{I}_{disc}$ and using
\begin{align*}
    W_i = f(Z_i, X_i) + \varphi_i 
\end{align*}
with $\varphi_i \sim \left(0, \sigma_{\varphi}^2\right)$. The conditional expected value of the treatment indicator is now defined as       
\begin{align*}
    \mathbb{E}\left[W_i | Z_i = z, X_i=x \right] = \delta(z, x).
\end{align*}
This conditional expectation is governed by a standard BART prior on $\delta(z, x)$ such that the denominator of Definition \ref{defn:cCACE} is estimated using BART in the sense of \cite{Hill_2011}. This gives us estimates $\widehat{\pi}_{C}(x)$ as required in discovery step 2(b) of Algorithm \ref{algo:BCF-IV} such that, together with $\widehat{ITT}_y(x)$, we can compute $\widehat\tau_{CACE}(x)$ straightforward.
Finally, a CART algorithm is used on the estimates $\widehat\tau_{CACE}(x)$ to discover heterogeneity in CACE while retaining interpretability by providing transparency on which variables have been used to construct the relevant subgroups similar to the sensitivity analysis approach conducted in \cite{hahn_bayesian_2020}. 

\subsection{Inference of conditional CACE}
\label{inference}
Finally, Definition \ref{defn:cCACE} is used in all subgroups independently using the tree structure learned in Subsection \ref{discovery} with unseen data from $\mathcal{I}_{inf}$ to infer conditional CACE by exploiting honest sample splitting outlined in Subsection \ref{honest_splitting}.
Following Definition \ref{defn:cCACE_estimator} and its corresponding conditionalized system of two simulatenous equations one can estimate $\widehat{\tau}_{2SLS, \mathbb{X}_j}$ by using 2SLS given a sufficient number of $i.i.d$ observations in each subgroup $\mathbb{X}_j$.  





\section{Shrinkage BCF-IV}
\label{sec:SBCF-IV}

The SBCF proposed by \cite{caron_shrinkage_2022} is a sparsity-inducing extension of the BCF which implements additional priors that uses weight-adjustments for covariates connected to the splits in the sum-of-tree model of the original BART algorithm \citep{chipman_bart_2010}. This extension suggests improved estimation of CATE for a rising number of covariates leading to a sparse data scenario as well was confounding data settings. 
The main difference between SBCF and BCF concerns the splitting probabilities that give each predictor variable a certain probabilty of being chosen as a splitting variable in the branching process of BART. 
For BCF, the vector of splitting probabilities $\vs=\left(s_1, ..., s_P\right)$ with $s_j$ being the probability for predictor variable $j \in \left(1, ..., P \right)$ of being chosen for a split is uniformly distributed with $s_j=\frac{1}{P} \forall j \in \{1, ..., P \}$. Instead, SBCF uses an additional Dirichlet prior on $\vs$ such that $\vs =\left(s_1, ..., s_P \right) \sim \text{Dirichlet}\left( \frac{\alpha}{P}, ..., \frac{\alpha}{P} \right)$. Therefore, small $\alpha$ values indicate a preference for sparsity while large $\alpha$ values would support many predictor variables in the model. As the degree of sparsity is typically unknown, an additional prior is used such that $\frac{\alpha}{\alpha+\rho} \sim {\text{Beta}\left(a, b\right)}$ with default values $\left(a, b, \rho \right) = \left(0.5, 1,P \right)$. That is, $(a, b) = (0.5, 1)$ leads to small $\alpha$ and with increasing number of predictors $P$ the preference for sparsity increases.\footnote{Moreover, $(a, b) = (1, 1)$ would result in the standard BART uniform distribution.}

SBCF-IV adapts the Discovery Step in the original Algorithm \ref{algo:BCF-IV} of BCF-IV in the following ways. First, in Step 2(a), BCF \citep{hahn_bayesian_2020} is replaced by SBCF for estimation of $\widehat{ITT}(x)$. 
Then, in Step 2(b), SBCF-IV uses a SoftBART probit algorithm to estimate $\widehat{\pi_C}(x)$ instead of BCF.\footnote{This is not yet implemented in the current Simulation Study.} 
Next, in Step 2(c), posterior splitting probabilities of the estimation in Step 2 (a) with SBCF are used in the \texttt{cost}-argument of \texttt{rpart} to scale the predictors when considering the splits such that the improvement of on splitting on some predictor is augmented by its cost in deciding which split to use \cite{therneau}.










\chapter{Simulation study}


Performance criteria according to bargagli-stoffi: 

1. Average number of truly discovered heterogeneous subgroups corresponding to the nodes of the generated CART (proportion of correctly discovered subgroups);

2. Monte Carlo estimated bias for the heterogeneous subgroups:
\begin{equation}
\text{Bias}_m(I_\text{inf}) = \frac{1}{N_\text{inf}} \sum_{i=1}^{N_\text{inf}} \sum_{l=1}^{L} \left( \tau_{\text{cace},i}(\ell) - \hat{\tau}_{\text{cace},i}(\ell, \Pi_m, I_\text{inf}) \right),
\end{equation}
\begin{equation}
\text{Bias}(I_\text{inf}) = \frac{1}{M} \sum_{m=1}^{M} \text{Bias}_m(I_\text{inf}),
\end{equation}
where $\Pi_m$ is the partition selected in simulation $m$, $L$ is the number of subgroups with heterogeneous effects (i.e., two for the case of strong heterogeneity and four for the case of slight heterogeneity), and $N_\text{inf}$ is the number of observations in the inference sample.

3. Monte Carlo estimated MSE for the heterogeneous subgroups:
\begin{equation}
\text{MSE}_m(I_\text{inf}) = \frac{1}{N_\text{inf}} \sum_{i=1}^{N_\text{inf}} \sum_{l=1}^{L} \left( \tau_{\text{cace},i}(\ell) - \hat{\tau}_{\text{cace},i}(\ell, \Pi_m, I_\text{inf}) \right)^2,
\end{equation}
\begin{equation}
\text{MSE}(I_\text{inf}) = \frac{1}{M} \sum_{m=1}^{M} \text{MSE}_m(I_\text{inf});
\end{equation}

4. Monte Carlo coverage, computed as the average proportion of units for which the estimated 95\% confidence interval of the causal effect in the assigned leaf includes the true value, for the heterogeneous subgroups:
\begin{equation}
C_m(I_\text{inf}) = \frac{1}{N_\text{inf}} \sum_{i=1}^{N_\text{inf}} \sum_{l=1}^{L} \left( \tau_{\text{cace},i}(\ell) \in \hat{\text{CI}}_{95} \left( \hat{\tau}_{\text{cace},i}(\ell, \Pi_m, I_\text{inf}) \right) \right),
\end{equation}
\begin{equation}
C(I_\text{inf}) = \frac{1}{M} \sum_{m=1}^{M} C_m(I_\text{inf}).
\end{equation}



\chapter{Empirical application}

\chapter{Discussion}

Moreover, \cite{Wooldridge2015} provide theorems for consistency and asymptotic normality for the unconditional TSLS estimator for the true population parameter $\tau_{CACE}$. \cite{bargagli-stoffi_heterogeneous_2022} show that those theorems can be transferred to the conditional TSLS estimator if there are sufficient number of observations for every subgroup in which the cCACE is estimated. 

\begin{thm}[Consistency and Asymptotic Normality of the Conditional 2SLS Estimator]
   Let Assumptions 1, 2, and 3 hold, i.e., \(E(Z^2_{i,X_j}) \neq 0\) (Assumption 1), \(E(Z_{i,X_j} \varepsilon_{i,X_j}) = 0\) (Assumption 2), and \(\pi_{C,X_j} \neq 0\) (Assumption 3). Then:
   \begin{enumerate}
       \item (Consistency) \( \hat{\tau}^{2SLS}_{X_j} - \tau_{X_j} \overset{p}{\to} 0 \) as \( N_{X_j} \to \infty \), where \( \overset{p}{\to} \) denotes convergence in probability, and \( N_{X_j} \) is the number of observations within the node \( X_j \).
       \item (Asymptotic Normality) If, in addition, \(E(Z^2_{i,X_j} \varepsilon^2_{i,X_j})\) is finite (Assumption 4), then:
       \[
       \sqrt{N_{X_j}} (\hat{\tau}^{2SLS}_{X_j} - \tau_{X_j}) \overset{d}{\to} \mathcal{N}(0, N_{X_j} \cdot \text{avar}(\hat{\tau}^{2SLS}_{X_j}))
       \]
       as \(N_{X_j} \to \infty\), where \( \overset{d}{\to} \) denotes convergence in distribution, \( \mathcal{N}(0, N_{X_j} \cdot \text{avar}(\hat{\tau}^{2SLS}_{X_j})) \) stands for the normal distribution, and \(\text{avar}(\hat{\tau}^{2SLS}_{X_j})\) is the asymptotic variance of the 2SLS estimator that can be approximated as in Chapter 15 of Wooldridge (2015).
   \end{enumerate}
\end{thm}











%\section{Appendices}
%\addcontentsline{toc}{section}{Appendices}



\Urlmuskip=0mu plus 1mu\relax
\bibliographystyle{agsm}
\bibliography{refs/BART_IV_paper.bib}

\cleardoubleemptypage



\end{document}