\chapter{Potential outcomes and irregular assignmnet}

We follow the setup of \cite{bargagli_stoffi_causal_2020} and \cite{bargagli-stoffi_heterogeneous_2022} who follow the usual notation of the Rubin's causal model. 

Given a set of $N$ indiviudals, indexed by $i = 1, \ldots, N$, we denote with $Y_i$ a generic outcome variable, with $W_i$ a binary treatment indicator, with $X$ an $N \times P$ matrix of $P$ control variables, and with $X_i$ the $i$-th $P$-dimensional row vector of covariates.
Let us define the pair of potential outcomes $Y_i(W_i)$ by using the Stable Unit Treatment Value Assumption (SUTVA). 
\begin{assump}{Stable unit treatment value assumption (SUTVA).}
   \label{assump:SUTVA} \\
   \begin{align*}
   \label{SUTVA}
   \text{If } W_i = w \text{, then } Y_i(w) = Y_i^{obs} \text{ , } \forall w \in \{0, 1 \} \text{ , } \forall i \in \{1, ..., N \}. 
   \end{align*}
\end{assump}
Therefore, it holds that $Y_i(W_i = 1) = Y_i(1)$ for an indiviudal $i$ under assignment to the treatment group and $Y_i(W_i = 0) = Y_i(0)$ under assignment to control group.
Despite one is unable to observe both potential outcomes at the same time for each individual one is able to observe the potential outcome that aligns with the assigned treatment status such that

\[
Y_i^{\text{obs}} = Y_i(1)W_i + Y_i(0)(1 - W_i).
\]

Under the strong ignorability assumptions of \ref{assump:unconfoundedness} and \ref{assump:overlap} we operate under the regular assignment mechanism which, through \ref{assump:unconfoundedness}, prevents the existence of unmeasured confounding and, by \ref{assump:overlap}, allows for unbiased treatment effect estimation in support of the covariate space. 

\begin{assump}{Unconfoundedness.}
   \label{assump:unconfoundedness} \\
   \begin{align*}
   W_i \ind \left(Y_i(1), Y_i(0)\right) |X_i), \text{ or equivalently, } \\
   Pr(W_i | (Y_i(1), Y_i(0), X_i) = Pr(W_i | X_i).   
   \end{align*}
\end{assump}

\begin{assump}{Positivity.}
   \label{assump:overlap} 
   \begin{align*}
   \epsilon < p(X_i=x) < 1 - \epsilon \text{ with probability 1, } \forall   \epsilon>0 , \forall x \text{ in support of } X_i.
   \end{align*}
\end{assump}

 
Using strong ignorability, one can define the CATE to analyse heterogeneous treatment effects by

\begin{defn}{Conditional Average Treatmnet Effect (CATE).}
   \label{defn:CATE} \\
   \begin{align*}
      \tau(x) = \mathbb{E}[Y_i^{\text{obs}} \mid W_i = 1, X_i = x] - \mathbb{E}[Y_i^{\text{obs}} \mid W_i = 0, X_i = x].   
   \end{align*}
\end{defn}

In this work, we deviate from this regular assignment mechanism that is implied by assumptions \ref{assump:unconfoundedness} and \ref{assump:overlap} and operate under the scenario of an irregular assignment mechanism.
This irregular assignment mechanism allows for a violation of compliance between (quasi-)randomized treatment assignment, $Z_i$, and treatment receipt, $W_i$, such that just the assignment to treatment is assumed to be unconfounded through a valid choice of $Z_i$ but the treatment receipt might be confounded.
A remedy to the issue of confoundedness in treatment receipt is the Instrumental Variable (IV) approach.
A proper instrumental variable, $Z_i$, affects treatment receipt, $W_i$, while not being allowed to affect $Y_i$ directly such that treatment receipt depends on the treatment assignment by $W_i(Z_i)$.
Based on the functional relation between $W_i$ and $Z_i$ and in case of a binary treatment variable, one can define four subgroups of individuals, $G_i$, such that  
\begin{defn}{Subgroups $G_i$.}
   \label{defn:subgroups}
   \begin{align*}
      G_i=
      \begin{cases}
         C,& \text{if } W_i(Z_i = 0) = 0, W_i(Z_i = 1) = 1 \\
         D,& \text{if } W_i(Z_i = 0) = 1, W_i(Z_i = 1) = 0 \\
         AT,& \text{if } W_i(Z_i = 0) = 1, W_i(Z_i = 1) = 1 \\
         NT,& \text{if }W_i(Z_i = 0) = 0, W_i(Z_i = 1) = 0
      \end{cases}.
   \end{align*}
\end{defn}
where $C$, $D$, $AT$ and $NT$ are abbreviations for Compliers, Defiers, Always-Takers, Never-Takers. 
The proportion of individuals that belong to each subgroup is defined as $\pi_{G_i}$, i.e. the proportion of compliers reads $\pi_{C}$.
Considering the distinction between $Z_i$ and $W_i$, the Intention-To-Treat (ITT) effect is defined as
\begin{defn}{Intention-To-Treat (ITT) effect.}
   \begin{align*}
      ITT_Y = \mathbb{E} \left[ Y_i | Z_i = 1 \right] - \mathbb{E} \left[ Y_i | Z_i = 0 \right],
   \end{align*}
\end{defn}
which refers to the instrument's average effect. 
Based on the subgroups in \ref{defn:subgroups} and the proportions $\pi_{G_i}$, an IV setting can be formalized by the usual four assumptions (\ref{assump:exclusion}, \ref{assump:monotonicity}, \ref{assump:existence}, \ref{assump:unconfoundedness_iv}) following \cite{Angrist1996}: 

\begin{assump}{Exclusion restriction.}
   \label{assump:exclusion} 
   \begin{align*}
      Y_i(0) = Y_i(1), \text{ for } G_i \in \{AT, NT \}.
   \end{align*}
\end{assump}
\begin{assump}{Monotonicity.}
   \label{assump:monotonicity}
   \begin{align*}
      W_i(1) \geq W_i(0) \rightarrow \pi_D = 0.
   \end{align*}
\end{assump}
\begin{assump}{Existence of compliers.}
   \label{assump:existence}
   \begin{align*}
      P(W_i(0) < W_i(1)) > 0 \rightarrow \pi_C \ne 0.
   \end{align*}
\end{assump}
\begin{assump}{Unconfoundedness of the instrumental variable.}
   \label{assump:unconfoundedness_iv}
   \begin{align*}
      Z_i \perp (Y_i(0, 0), Y_i(0, 1), Y_i(1, 0), Y_i(1, 1), W_i(0), W_i(1)).
   \end{align*}
\end{assump}

If these assumptions \ref{assump:exclusion} - \ref{assump:unconfoundedness_iv} hold, the Complier Average Causal Effect (CACE) is identified as

\begin{defn}{Complier Average Causal Effect (CACE)}
   \label{defn:CACE}
   \begin{align*}
      \tau_{\text{CACE}} = \frac{\text{ITT}_{Y}}{\pi_C} = \frac{\mathbb{E}[Y_i \mid Z_i = 1] - \mathbb{E}[Y_i \mid Z_i = 0]}{\mathbb{E}[W_i \mid Z_i = 1] - \mathbb{E}[W_i \mid Z_i = 0]}.
   \end{align*}
\end{defn}

can be estimated from observational data. The numerator represents the average effect of the instrument, also referred to as the Intention-To-Treat (ITT) effect. 
The denominator represents the overall proportion of indiviudals that comply with the treatment assignment, also referred to as the proportion of compliers \cite{Angrist1996}.
CACE is also sometimes referred to as Local Average Treatment Effects (see \cite{AngristPischke2008}) and represents the estimate of the causal effect of the assignment to treatment on the principal outcome, $Y_i$, for the subpopulation of compliers \cite{ImbensRubin2015}.



Following \cite{bargagli-stoffi_heterogeneous_2022}, we consider the following conditional version of the CACE, 

\begin{defn}{Conditional CACE (cCACE).}
   \begin{align*}
      \tau_{\text{CACE}}(x) = \frac{\text{ITT}_{Y}(x)}{\pi_C(x)} = \frac{\mathbb{E}[Y_i \mid Z_i = 1, X_i = x] - \mathbb{E}[Y_i \mid Z_i = 0, X_i = x]}{\mathbb{E}[W_i \mid Z_i = 1, X_i = x] - \mathbb{E}[W_i \mid Z_i = 0, X_i = x]}.
   \end{align*}
\end{defn}

The cCACE is a straightforward extension of the CACE in \ref{defn:CACE} with \cite{bargagli-stoffi_heterogeneous_2022} providing theorems of a consistent and asymptotic normal conditional Two-Stage-Least-Square (2SLS) estimator which rely on the unconditional theorems of Wooldridge. 
The conditional theorems just need the additional assumption of sufficient number of observations for every leaf node in which the cCACE is estimated.








The conditional CACE can be estimated in a generic sub-sample (i.e., for each $\mathbf{X}_i \in \mathbb{X}_j$ , where $\mathbb{X}_j$ is a
generic node of the discovered tree, like a non-terminal node or a leaf) as: