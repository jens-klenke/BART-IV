\chapter{Potential outcomes and irregular assignmnet}

We follow the setup of \cite{bargagli_stoffi_causal_2020} and \cite{bargagli-stoffi_heterogeneous_2022} who use the usual notation of the Rubin's causal model. Given a set of $N$ indiviudals, indexed by $i = 1, \ldots, N$, we denote with $Y_i$ a generic outcome variable, with $W_i$ a binary treatment indicator, with $X$ an $N \times P$ matrix of $P$ control variables, and with $X_i$ the $i$-th $P$-dimensional row vector of covariates.
Let us define the pair of potential outcomes $Y_i(W_i)$ by using the Stable Unit Treatment Value Assumption (SUTVA). 
\begin{assump}{Stable unit treatment value assumption (SUTVA).}
   \label{assump:SUTVA} \\
   \begin{align*}
   \label{SUTVA}
   \text{If } W_i = w \text{, then } Y_i(w) = Y_i^{obs} \text{ , } \forall w \in \{0, 1 \} \text{ , } \forall i \in \{1, ..., N \}. 
   \end{align*}
\end{assump}
Therefore, it holds that $Y_i(W_i = 1) = Y_i(1)$ for an indiviudal $i$ under assignment to the treatment group and $Y_i(W_i = 0) = Y_i(0)$ under assignment to control group.
Despite one is unable to observe both potential outcomes at the same time for each individual one is able to observe the potential outcome that aligns with the assigned treatment status such that

\[
Y_i^{\text{obs}} = Y_i(1)W_i + Y_i(0)(1 - W_i).
\]

Under the following strong ignorability assumptions of Assumption \ref{assump:unconfoundedness} and Assumption \ref{assump:overlap} we operate under the regular assignment mechanism which, through Assumption \ref{assump:unconfoundedness}, prevents the existence of unmeasured confounding and, by Assumption \ref{assump:overlap}, allows for unbiased treatment effect estimation in support of the covariate space. 

\begin{assump}{Unconfoundedness.}
   \label{assump:unconfoundedness} \\
   \begin{align*}
   W_i \ind \left(Y_i(1), Y_i(0)\right) |X_i), \text{ or equivalently, } \\
   Pr(W_i | (Y_i(1), Y_i(0), X_i) = Pr(W_i | X_i).   
   \end{align*}
\end{assump}

\begin{assump}{Positivity.}
   \label{assump:overlap} 
   \begin{align*}
   \epsilon < p(X_i=x) < 1 - \epsilon \text{ with probability 1, } \forall   \epsilon>0 , \forall x \text{ in support of } X_i.
   \end{align*}
\end{assump}

 
Using strong ignorability, one can define, for instance, the CATE to analyze heterogeneous treatment effects by

\begin{defn}{Conditional Average Treatmnet Effect (CATE).}
   \label{defn:CATE} \\
   \begin{align*}
      \tau(x) = \mathbb{E}[Y_i^{\text{obs}} \mid W_i = 1, X_i = x] - \mathbb{E}[Y_i^{\text{obs}} \mid W_i = 0, X_i = x].   
   \end{align*}
\end{defn}

In this work, we deviate from this regular assignment mechanism that is implied by Assumptions \ref{assump:unconfoundedness} and \ref{assump:overlap} and operate under the scenario of an irregular assignment mechanism.
This irregular assignment mechanism allows for a violation of compliance between (quasi-)randomized treatment assignment, $Z_i$, and treatment receipt, $W_i$, such that just the assignment to treatment is assumed to be unconfounded through a valid choice of $Z_i$ but the treatment receipt might be confounded.
A remedy to the issue of confoundedness in treatment receipt is the Instrumental Variable (IV) approach.
A proper instrumental variable, $Z_i$, affects treatment receipt, $W_i$, while not being allowed to affect $Y_i$ directly such that treatment receipt depends on the treatment assignment by $W_i(Z_i)$.
Based on the functional relation between $W_i$ and $Z_i$ and in case of a binary treatment variable, one can define four subgroups of individuals, $G_i$, such that  
\begin{defn}{Subgroups $G_i$.}
   \label{defn:subgroups}
   \begin{align*}
      G_i=
      \begin{cases}
         C,& \text{if } W_i(Z_i = 0) = 0, W_i(Z_i = 1) = 1 \\
         D,& \text{if } W_i(Z_i = 0) = 1, W_i(Z_i = 1) = 0 \\
         AT,& \text{if } W_i(Z_i = 0) = 1, W_i(Z_i = 1) = 1 \\
         NT,& \text{if }W_i(Z_i = 0) = 0, W_i(Z_i = 1) = 0
      \end{cases}.
   \end{align*}
\end{defn}
where $C$, $D$, $AT$ and $NT$ are abbreviations for Compliers, Defiers, Always-Takers, Never-Takers. 
The proportion of individuals that belong to each subgroup is defined as $\pi_{G_i}$, i.e. the proportion of compliers reads $\pi_{C}$.
Considering the distinction between $Z_i$ and $W_i$, the Intention-To-Treat (ITT) effect is defined as
\begin{defn}{Intention-To-Treat (ITT) effect.}
   \begin{align*}
      ITT_Y = \mathbb{E} \left[ Y_i | Z_i = 1 \right] - \mathbb{E} \left[ Y_i | Z_i = 0 \right].
   \end{align*}
\end{defn}
The ITT effect refers to the instrument's average effect. 
Based on the subgroups in Definition \ref{defn:subgroups} and their proportions $\pi_{G_i}$, an IV setting can be formalized by the usual four IV assumptions following \cite{Angrist1996}.

\begin{assump}{Classical IV assumptions with a binary treatment.}
   \label{assump:IV_assump} 
   \begin{align*}
      &\text{Exclusion restriction: }& Y_i(0) = Y_i(1), \text{ for } G_i \in \{AT, NT \}. \\
      &\text{Monotonicity: }& W_i(1) \geq W_i(0) \rightarrow \pi_D = 0. \\
      &\text{Existence of compliers: }& P(W_i(0) < W_i(1)) > 0 \rightarrow \pi_C \ne 0. \\
      &\text{Unconfoundedness of IV: }& Z_i \perp (Y_i(0, 0), Y_i(0, 1), Y_i(1, 0), Y_i(1, 1), W_i(0), W_i(1)). 
   \end{align*}
\end{assump}

If Assumptions \ref{assump:IV_assump} hold, the Complier Average Causal Effect (CACE) can be identified.

\begin{defn}{Complier Average Causal Effect (CACE).}
   \label{defn:CACE}
   \begin{align*}
      \tau_{\text{CACE}} = \frac{\text{ITT}_{Y}}{\pi_C} = \frac{\mathbb{E}[Y_i \mid Z_i = 1] - \mathbb{E}[Y_i \mid Z_i = 0]}{\mathbb{E}[W_i \mid Z_i = 1] - \mathbb{E}[W_i \mid Z_i = 0]},
   \end{align*}
\end{defn}

The CACE can be estimated from observational data where the numerator represents the average effect of the instrument, also referred to as the Intention-To-Treat (ITT) effect. The denominator represents the overall proportion of indiviudals that comply with the treatment assignment, also referred to as the proportion of compliers \cite{Angrist1996}.
CACE is also sometimes referred to as Local Average Treatment Effects (LATE, see \cite{AngristPischke2008}) and represents the estimate of the causal effect of the assignment to treatment on the principal outcome, $Y_i$, for the subpopulation of compliers \cite{ImbensRubin2015}. Consider the system of two simultaneous equations 
\begin{align*}
Y_i^{obs} &= \alpha + \tau_{CACE} W_i + \epsilon, \\
W_i &= \pi_0 + \pi_C Z_i + \eta_i,
\end{align*}
with $\mathbb{E}\left(\epsilon_i \right) = \mathbb{E}\left(\eta_i \right) = 0$ and we assume by the first equation a linear projection of $W_i$ onto $Z_i$ with $\mathbb{E}\left(Z_i \eta_i\right)=0$. Then, \cite{Angrist1996} and \cite{ImbensRubin2015} show that $\tau_{\text{CACE}}$ can be estimated by a Two Stage Least Square (TSLS) estimator which is consistent and asymptotic normal as displayed in \cite{Wooldridge2015}.   

In this work, we follow \cite{bargagli-stoffi_heterogeneous_2022} and consider the conditional version of the CACE, 

\begin{defn}{Conditional CACE (cCACE).}
   \begin{align*}
      \tau_{\text{CACE}}(x) = \frac{\text{ITT}_{Y}(x)}{\pi_C(x)} &= \frac{\mathbb{E}[Y_i \mid Z_i = 1, X_i = x] - \mathbb{E}[Y_i \mid Z_i = 0, X_i = x]}{\mathbb{E}[W_i \mid Z_i = 1, X_i = x] - \mathbb{E}[W_i \mid Z_i = 0, X_i = x]}.
   \end{align*}
\end{defn}

The cCACE is a straightforward extension of the CACE in Definition \ref{defn:CACE} presented in \cite{bargagli-stoffi_heterogeneous_2022}. A natural, subgroup-related estimator $\tau_{\text{CACE}}(x)$ can be defined by acknowledging $\mX_i \in \mathbb{X}_j$ with $\mathbb{X}_j$ being a pre-specified subgroup.

\begin{defn}{Estimator of cCACE.}
   \label{defn:cCACE_estimator}
   \begin{align*}
      \widehat\tau_{\text{CACE}}(x) &= \frac{\widehat{\text{ITT}}_{Y}(x)}{\widehat{\pi}_C(x)} \\
      &= \frac{\frac{1}{N_{1,j}} \mathlarger{\sum}_{l: X_l \in \mathbb{X}_j} Y^{obs}_l Z_l - \frac{1}{N_{0,j}} \mathlarger{\sum}_{l: X_l \in \mathbb{X}_j} Y^{obs}_l (1 - Z_l)}{\frac{1}{N_{1,j}} \mathlarger{\sum}_{l: X_l \in \mathbb{X}_j} W_l Z_l - \frac{1}{N_{0,j}} \mathlarger{\sum}_{l: X_l \in \mathbb{X}_j} W_l (1 - Z_l)
      }\\
   \end{align*}
\end{defn}

Intuitively, Definition \ref{defn:cCACE_estimator} implies to use TSLS subgroup-wise for every $\mathbb{X}_j$ under Assumptions \ref{assump:IV_assump}. The system of two simulatenous equations from above can be conditionalized by 
\begin{align*}
   Y_{i, \mathbb{X}_j}^{obs} &= \alpha_{\mathbb{X}_j} + \tau_{\mathbb{X}_j}^{CACE} W_{i, \mathbb{X}_j} + \epsilon_{i, \mathbb{X}_j}, \\
   W_{i, \mathbb{X}_j} &= \pi_{0, \mathbb{X}_j} + \pi_{C, \mathbb{X}_j} Z_{i, \mathbb{X}_j} + \eta_{i, \mathbb{X}_j},
\end{align*}
such that the reduced form reads 
\begin{align*}
   Y_{i,\mathbb{X}_j}^{\text{obs}} = &\left( \alpha_{\mathbb{X}_j} +
    \tau_{\text{CACE},\mathbb{X}_j} \pi_{0,\mathbb{X}_j} \right) + \\
     &\left( \tau_{\text{CACE},\mathbb{X}_j} \pi_{C,\mathbb{X}_j} \right) Z_{i,\mathbb{X}_j} + \\
     &\left( \varepsilon_{i,\mathbb{X}_j} + \tau^{\text{CACE}}_{\mathbb{X}_j} \eta_{i,\mathbb{X}_j} \right).
\end{align*}
such that the intercept and slope parameter can be estimated by ordinary least squares, given a sufficient number of $i.i.d$ observations in each node. More information regarding theoretical properties of ... can be found in Appendix A of \cite{bargagli-stoffi_heterogeneous_2022}. In Definition \ref{defn:cCACE_estimator}, the estimator $\widehat\tau_{\text{CACE}}(x)$ is defined by relying on the existence of accurately pre-specified subgroups. The main contribution of BCF-IV in \cite{bargagli-stoffi_heterogeneous_2022} revolves around providing a full algorithm that (1) $\mathcal{I}_{disc}, \mathcal{I}_{inf}$ and (2) discovers heterogeneity in cCACE in an interpretable way using $\mathcal{I}_{disc}$ and (3) infers precise estimates of cCACE on $\mathcal{I}_{inf}$. The next chapter explains BCF-IV in detail and describes the extension to a sparsity-inducing version.




