\chapter{Simulation Study}


\section{Setup}

Let us generate the set of potential outcomes, potential treatments and covariates for each individual $i$ by  

\begin{align*}
    Z_i &\sim \text{Binom}(0.5), \\
    W_i(Z_i=1) &\sim \text{Binom}(\pi_{comp}), \\
    W_i(Z_i=0) &= 0, \\
    W^{obs}_i &= Z_i W_i(1) +  (1-Z_i) W_i(0), \\
    Y_i(Z_i=0) &\sim \mathcal{N}(0, 1), \\
    Y_i(Z_i=1) &= Y_i(0) + \mu(X_i) + W_i(1) \tau^{CACE}(X_i).
\end{align*}
Let $i=1, ..., N \text{ with } N = 1000$ be the number of observations. The covariates are binary as well as continuous variables generated by 
\begin{align*}
    X_{i,1}, ..., X_{i, P/2} &\sim \text{Binom}(0.5), \\
    X_{i, P/2+1}, ..., X_{i, P} &\sim \mathcal{N}(0, 1), 
\end{align*}
with the number of covariates $P \in \{10, 50, 100\}$ to illustrate different degrees of sparsity.
Moreover, we differ between different strengths of compliance rates with $\pi_{comp} \in \{0.5, 0.75\}$ such that we discriminate between a weak and a strong instrument scenario. With $\pi_{comp} << 1$, we guarantee an operation under imperfect compliance where individuals may not comply with their actual treatment assignment. More precisely, we use one-sided non-compliance here, due to $W_i(0)=0$, as the potential treatment receipt of individuals that are not assigend to treatment is always zero. The effect size differs by  $k \in \{1 ,2\}$ such that we can compare a small and a greater amount of heterogeneity between subgroups.
We generate the following simplistic scenario for the heterogeneous conditional CACE by conditioning $\tau^{CACE}(X_i)$ solely on the binary variables $X_{i, 1}$ and $X_{i, 2}$ using 
\begin{align*} 
    \tau^{CACE}(X_i) = \begin{cases*}
        k \text{, if } X_i \in l_1 = \{X_i: X_{i,1}=0, X_{i,2}=0 \}, \\
        -k \text{, if } X_i \in l_2 = \{X_i: X_{i,1}=1, X_{i, 2}=1 \}, \\
        0 \text{, else.}
    \end{cases*}
\end{align*}

The baseline effect $\mu(.)$ solely depends on continuous covariates and is either unconfounded by 
\begin{align*}
    \mu(X_i) = 3 + 1.5 \sin{\left(\pi X_{i, P/2 + 1}\right)} + 0.5\left(X_{i, P/2+2} - 0.5\right)^2 + 
    1.5\left(2-\abs{X_{i, P/2+3}}\right),
\end{align*}
or confounded by extending to
\begin{align*}
    \mu(X_i) = &3 + 1.5 \sin{\left(\pi X_{i, P/2 + 1}\right)} + 0.5\left(X_{i, P/2+2} - 0.5\right)^2 + 
    1.5\left(2-\abs{X_{i, P/2+3}}\right) + \\
    &1.5X_{i, P/2 + 4}\left(X_{i, 1} + 1\right),
\end{align*}
where the heterogeneity-inducing covariate $X_{i, 1}$ (used within the determination of $\tau^{CACE}(X_i)$) is also present within $\mu(.)$.


\section{Performance Criteria}

Performance criteria according to bargagli-stoffi: 

1. Average number of truly discovered heterogeneous subgroups corresponding to the nodes of the generated CART (proportion of correctly discovered subgroups);

2. Monte Carlo estimated bias for the heterogeneous subgroups:
\begin{equation}
\text{Bias}_m(I_\text{inf}) = \frac{1}{N_\text{inf}} \sum_{i=1}^{N_\text{inf}} \sum_{l=1}^{L} \left( \tau_{\text{cace},i}(\ell) - \hat{\tau}_{\text{cace},i}(\ell, \Pi_m, I_\text{inf}) \right),
\end{equation}
\begin{equation}
\text{Bias}(I_\text{inf}) = \frac{1}{M} \sum_{m=1}^{M} \text{Bias}_m(I_\text{inf}),
\end{equation}
where $\Pi_m$ is the partition selected in simulation $m$, $L$ is the number of subgroups with heterogeneous effects (i.e., two for the case of strong heterogeneity and four for the case of slight heterogeneity), and $N_\text{inf}$ is the number of observations in the inference sample.

3. Monte Carlo estimated MSE for the heterogeneous subgroups:
\begin{equation}
\text{MSE}_m(I_\text{inf}) = \frac{1}{N_\text{inf}} \sum_{i=1}^{N_\text{inf}} \sum_{l=1}^{L} \left( \tau_{\text{cace},i}(\ell) - \hat{\tau}_{\text{cace},i}(\ell, \Pi_m, I_\text{inf}) \right)^2,
\end{equation}
\begin{equation}
\text{MSE}(I_\text{inf}) = \frac{1}{M} \sum_{m=1}^{M} \text{MSE}_m(I_\text{inf});
\end{equation}

4. Monte Carlo coverage, computed as the average proportion of units for which the estimated 95\% confidence interval of the causal effect in the assigned leaf includes the true value, for the heterogeneous subgroups:
\begin{equation}
C_m(I_\text{inf}) = \frac{1}{N_\text{inf}} \sum_{i=1}^{N_\text{inf}} \sum_{l=1}^{L} \left( \tau_{\text{cace},i}(\ell) \in \hat{\text{CI}}_{95} \left( \hat{\tau}_{\text{cace},i}(\ell, \Pi_m, I_\text{inf}) \right) \right),
\end{equation}
\begin{equation}
C(I_\text{inf}) = \frac{1}{M} \sum_{m=1}^{M} C_m(I_\text{inf}).
\end{equation}




\section{Results}
  
 

